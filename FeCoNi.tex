\documentclass[hyperref=unicode,presentation,10pt]{beamer}

\usepackage[absolute,overlay]{textpos}
\usepackage{array}
\usepackage{graphicx}
\usepackage{adjustbox}
\usepackage[version=4]{mhchem}
\usepackage{chemfig}
\usepackage{caption}
\usepackage{makecell}

%dělení slov
\usepackage{ragged2e}
\let\raggedright=\RaggedRight
%konec dělení slov

\addtobeamertemplate{frametitle}{
	\let\insertframetitle\insertsectionhead}{}
\addtobeamertemplate{frametitle}{
	\let\insertframesubtitle\insertsubsectionhead}{}

\makeatletter
\CheckCommand*\beamer@checkframetitle{\@ifnextchar\bgroup\beamer@inlineframetitle{}}
\renewcommand*\beamer@checkframetitle{\global\let\beamer@frametitle\relax\@ifnextchar\bgroup\beamer@inlineframetitle{}}
\makeatother
\setbeamercolor{section in toc}{fg=red}
\setbeamertemplate{section in toc shaded}[default][100]

\usepackage{fontspec}
\usepackage{unicode-math}

\usepackage{polyglossia}
\setdefaultlanguage{czech}

\def\uv#1{„#1“}

\mode<presentation>{\usetheme{default}}
\usecolortheme{crane}

\setbeamertemplate{footline}[frame number]

\title[Crisis]
{C2062 -- Anorganická chemie II}

\subtitle{Železo, kobalt a nikl}
\author{Zdeněk Moravec, hugo@chemi.muni.cz \\ \adjincludegraphics[height=60mm]{img/IUPAC_PSP.jpg}}
\date{}

\begin{document}

\begin{frame}
	\titlepage
\end{frame}

\frame{
	\frametitle{}
	\vfill
	\begin{center}
		\adjincludegraphics[width=\textwidth]{img/Periodic_table_AH.png}
	\end{center}
	\vfill
}

\section{Úvod}
\frame{
	\frametitle{}
	\vfill
	\begin{tabular}{|c|l|l|l|}
	\hline
	 & \textit{Železo} & \textit{Kobalt} & \textit{Nikl} \\\hline
	 El. konfigurace & 3d$^{6}$ 4s$^{2}$ & 3d$^{7}$ 4s$^{2}$ & 3d$^{8}$ 4s$^{2}$ \\\hline
	 Teplota tání [$^\circ$C] & 1538 & 1495 & 1455 \\\hline
	 Teplota varu [$^\circ$C]  & 2861 & 2927 & 2730 \\\hline
	 Objeven & pravěk & 1735 & 1751 \\\hline
	 Vzhled & leskle kovový\footnote[frame]{Zdroj: \href{https://commons.wikimedia.org/wiki/File:Iron_electrolytic_and_1cm3_cube.jpg}{Alchemist-hp/Commons}} & šedý\footnote[frame]{Zdroj: \href{https://commons.wikimedia.org/wiki/File:Cobalt-3.jpg}{Materialscientist/Commons}} & stříbrný\footnote[frame]{Zdroj: \href{https://commons.wikimedia.org/wiki/File:Nickel_kugeln.jpg}{René Rausch/Commons}} \\
	 &  \begin{minipage}{.2\textwidth}
	 	\adjincludegraphics[width=\linewidth]{img/Iron.jpg}
	 \end{minipage}
	 	& \begin{minipage}{.2\textwidth}
	 		\adjincludegraphics[width=\linewidth]{img/Cobalt-3.jpg}
	 	\end{minipage} & \begin{minipage}{.2\textwidth}
	 	\adjincludegraphics[width=\linewidth]{img/Nickel_kugeln.jpg}
 	\end{minipage} \\\hline
	\end{tabular}
	\vfill
}

\section{Historie}
\frame{
	\frametitle{}
	\vfill
	\begin{figure}
		\adjincludegraphics[width=\textwidth]{img/Long_Waves_of_Social_Evolution.jpg}
		\caption*{Technologický vývoj lidstva.\footnote[frame]{Zdroj: \href{https://commons.wikimedia.org/wiki/File:Long_Waves_of_Social_Evolution.jpg}{Myworkforwiki/Commons}}}
	\end{figure}
	\vfill
}

\subsection{Doba kamenná}
\frame{
	\frametitle{}
	\begin{columns}
		\begin{column}{0.6\textwidth}
			\vfill
			\begin{itemize}
				\item 3 milióny let př. n. l -- 3~000 let př.~n.~l.
				\item Člověk využíval přírodní materiály.
				\item Hlavním materiálem byl kámen -- pazourek, buližník.
				\item Lidé si začali stavět pevné příbytky a vyrábět pálené hliněné nádoby.
				\item Mimo kamene využíval člověk další dostupné přírodní materiály -- kosti, dřevo, parohy, \ldots
			\end{itemize}
			\vfill
		\end{column}
		\begin{column}{0.4\textwidth}
			\begin{figure}
				\adjincludegraphics[width=\textwidth]{img/Fragments_of_idols.jpg}
				\caption*{Fragmenty terakotových sošek, cca 5000 př. n. l.\footnote[frame]{Zdroj: \href{https://commons.wikimedia.org/wiki/File:Fragments_of_idols,_clay,_Sicily,_5000_BC,_AM_Syracuse,_121277.jpg}{Zde/Commons}}}
			\end{figure}
		\end{column}
	\end{columns}
}

\subsection{Rozvoj metalurgie}
\frame{
	\frametitle{}
	\begin{columns}
		\begin{column}{0.7\textwidth}
			\vfill
			\begin{itemize}
				\item Prvním kovem, který člověk zpracovával bylo pravděpodobně zlato (40~000~př.~n.~l.), které v přírodě nacházíme v elementárním stavu.
				\item Olovo, cín a měď lze získat z jejich rud zahřátím, proto patřily k prvním masověji zpracovávaným kovům.
				\item Slitina mědi s cínem -- bronz -- dala jméno době bronzové a umožnila výrazný technologický posun. S bronzem se setkáváme už okolo 6~000~př.~n.~l.
				\item Výroba železa z železné rudy je podstatně náročnější, proto byla tato technologie vyvinuta až v období okolo 1200~př.~n.~l.
			\end{itemize}
			\vfill
		\end{column}
		\begin{column}{0.35\textwidth}
			\begin{figure}
				\adjincludegraphics[width=\textwidth]{img/UlsterMuseumPrehistoryBrGoldLunulae.jpg}
				\caption*{Zlatá lunula z doby bronzové.\footnote[frame]{Zdroj: \href{https://commons.wikimedia.org/wiki/File:UlsterMuseumPrehistoryBrGoldLunulae_(2).JPG}{Notafly/Commons}}}
			\end{figure}
		\end{column}
	\end{columns}
}

\subsection{Doba bronzová}
\frame{
	\frametitle{}
	\begin{columns}
		\begin{column}{0.8\textwidth}
			\vfill
			\begin{itemize}
				\item 3000 -- 600 př. n. l.
				\item \textit{Bronz} -- slitina mědi, cínu a příp. i~dalšího kovu.
				\item Nejprve byla zpracovávána samotná měď, později bylo zjištěno, že přídavkem cínu lze získat slitinu, která je tvrdší a lépe se odlévá.
				\item Oba kovy tají za nižší teploty než železo (1538~$^{\circ}$C), proto bylo jejich zpracování snazší.
				\item V této době došlo k vývoji nové techniky -- odlévání kovů do forem a používání nýtů.
				\item Bronz byl pravděpodobně znám dříve než čistý cín.
			\end{itemize}
			\vfill
		\end{column}
		\begin{column}{0.3\textwidth}
			\begin{figure}
				\adjincludegraphics[height=0.6\textheight]{img/Dagger_Early_Bronze_Age.jpg}
				\caption*{Trojúhelníková dýka ze starší doby bronzové.\footnote[frame]{Zdroj: \href{https://commons.wikimedia.org/wiki/File:Dagger_Early_Bronze_Age.jpg}{José-Manuel Benito Álvarez/Commons}}}
			\end{figure}
		\end{column}
	\end{columns}
}

\subsection{Doba železná}
\frame{
	\frametitle{}
	\vfill
	\begin{itemize}
		\item Výroba železa je technologicky náročnější než mědi a~cínu. Železo je ale výhodnější díky vyšší tvrdosti a~dostupnosti jeho rud.
		\item Prvními zdroji byly zbytky meteoritů (sideritů), které obsahovaly slitinu železa s~niklem.
		\item Znalost výroby železa se rozšířila z Malé Asie kolem poloviny 2.~tisíciletí.
		\item Existují důkazy, že se ocel v Evropě vyráběla již před našim letopočtem.\footnote[frame]{\href{https://www.osel.cz/12757-usvit-technologii-v-evrope-se-pouzivala-kvalitni-ocel-jiz-pred-2-900-lety.html}{Úsvit technologií: V Evropě se používala kvalitní ocel již před 2 900 lety}}
		\item Nejprve se železo vyrábělo přímo z rud v pecích vytápěných dřevěným uhlím. Takto připravené železo bylo pórovité, ale dobře kujné. Označovalo se jako \textit{svářková ocel} a obsahovalo jen malý podíl uhlíku.
	\end{itemize}
	\vfill
}

\frame{
	\frametitle{}
	\begin{columns}
		\begin{column}{0.5\textwidth}
			\begin{figure}
				\adjincludegraphics[width=\textwidth]{img/Campo-iron-meteorite.jpg}
				\caption*{Železný meteorit, 576 g.\footnote[frame]{Zdroj: \href{https://commons.wikimedia.org/wiki/File:Campo-iron-meteorite.jpg}{Geoking42 /Commons}}}
			\end{figure}
		\end{column}
		\begin{column}{0.5\textwidth}
			\begin{figure}
				\adjincludegraphics[height=.6\textheight]{img/Bas_fourneau.png}
				\caption*{Výroba železa ve středověku.\footnote[frame]{Zdroj: \href{https://commons.wikimedia.org/wiki/File:Bas_fourneau.png}{Helix84/Commons}}}
			\end{figure}
		\end{column}
	\end{columns}
}

\subsection{Keramika}
\frame{
	\frametitle{}
	\begin{columns}
		\begin{column}{0.8\textwidth}
			\vfill
			\begin{itemize}
				\item Prvními keramickými výrobky byly hliněné mísy sušené na slunci.
				\item Pálená keramika je známá až od cca 10 000 let př.~n.~l.
				\item Ve 3. tisíciletí př. n. l. byl vynalezen tzv. \textit{rychlý hrnčířský kruh}, který nahradil kruh roztáčený rukou.
				\item Ve 2. tisíciletí př. n. l. je keramika využívána ve stavebnictví, dochází k rozvoji cihlářství.
				\item V letech 600--900 se v Číně začíná s výrobou bílého porcelánu.
			\end{itemize}
			\vfill
		\end{column}
		\begin{column}{0.3\textwidth}
			\begin{figure}
				\adjincludegraphics[height=0.55\textheight]{img/Xuande_Archaic_Porcelain_Vase.jpg}
				\caption*{Čínská porcelánová váza.\footnote[frame]{Zdroj: \href{https://commons.wikimedia.org/wiki/File:Xuande_Archaic_Porcelain_Vase.JPG}{King muh/Commons}}}
			\end{figure}
		\end{column}
	\end{columns}
}

\subsection{Sklo}
\frame{
	\frametitle{}
	\begin{columns}
		\begin{column}{0.8\textwidth}
			\vfill
			\begin{itemize}
				\item Přírodní sklo (obsidián), bylo využíváno už v době kamenné.
				\item Historie výroby skla začíná už v době bronzové a souvisí s vývojem keramiky.
				\item První skla byla barevná nebo černá, čiré sklo se objevuje až později.
				\item V 5. století př. n. l. se začínají využívat formy
				\item Technika foukání skla byla objevena v 1. století př.~n.~l.
				\item Pece na zpracování skla byly vytápěny spalováním dřeva, v současnosti se téměř výhradně používá plyn.
				\item Počátky českého sklářství jsou datovány o přelomu 12. a 13.~století.
			\end{itemize}
			\vfill
		\end{column}
		\begin{column}{0.3\textwidth}
			\begin{figure}
				\adjincludegraphics[width=\textwidth]{img/Obsidian.jpg}
				\caption*{Obsidián.\footnote[frame]{Zdroj: \href{https://commons.wikimedia.org/wiki/File:Obsidian_1.jpg}{Karelj/Commons}}}
			\end{figure}
		\end{column}
	\end{columns}
}

\section{Chemické a fyzikální vlastnosti}
\subsection{Železo}
\frame{
	\frametitle{}
	\vfill
	\begin{columns}
		\begin{column}{.5\textwidth}
			\begin{itemize}
		\item V čistém stavu není příliš pevné, ale dá se dobře opracovávat.
		\item Vytváří čtyři allotropní modifikace:
		\begin{itemize}
			\item $\alpha$-Fe -- BCC
			\item $\gamma$-Fe -- FCC
			\item $\delta$-Fe -- BCC
			\item $\epsilon$-Fe - vysokoteplotní modifikace, nad 10~GPa. Nejtěsnější hexagonální uspořádání.\footnote[frame]{\href{https://www.jstor.org/stable/1714581?seq=1\#metadata_info_tab_contents}{High-Pressure Polymorph of Iron}}
		\end{itemize}
		\item Fyzikální i magnetické vlastnosti jsou závislé na čistotě železa.
	\end{itemize}
	\end{column}
	\begin{column}{.6\textwidth}
	\begin{figure}
		\adjincludegraphics[width=\textwidth]{img/Pure_iron_phase_diagram.png}
		\caption*{Fázový diagram železa.\footnote[frame]{Zdroj: \href{https://commons.wikimedia.org/wiki/File:Pure_iron_phase_diagram_(EN).svg}{Daniele Pugliesi/Commons}}}
	\end{figure}
	\end{column}
	\end{columns}
	\vfill
}

\frame{
	\frametitle{}
	\vfill
	\begin{itemize}
		\item Čisté \textit{železo} je do teploty 768~$^\circ$C (Curieho teplota\footnote[frame]{Nad Curieovou teplotou ztrácí látka své feromagnetické (či piezoelektrické) vlastnosti.}) feromagnetické.
		\item V práškovém stavu je pyroforické, v bulku se oxiduje vzduchem až za vyšší teploty.
		\item Ochotně se rozpouští ve zředěných kyselinách za vzniku železnatých solí. Oxidující kyseliny způsobují pasivaci povrchu.
		\item Vytváří sloučeniny v oxidačních stavech II, III a VI.
		\item Slučuje se s většinou nekovů, ochotně se oxiduje kyslíkem, zvláště ve vlhkém prostředí.
		\item Má čtyři stabilní izotopy a 24 radioizotopů:\footnote[frame]{\href{https://www.webelements.com/iron/isotopes.html}{Iron: isotope data}}
	\end{itemize}
	\begin{tabular}{|l|r@{,}l|}
		\hline
		\ce{^{54}Fe} & 5 & 845 \% \\\hline
		\ce{^{56}Fe} & 91 & 754 \% \\\hline
		\ce{^{57}Fe} & 2 & 119 \% \\\hline
		\ce{^{58}Fe} & 0 & 282 \% \\\hline
	\end{tabular}
	\vfill
}

\subsection{Kobalt}
\frame{
	\frametitle{}
	\vfill
	\begin{itemize}
		\item \textit{Kobalt} má stříbrnou barvu s jemným modrošedým nádechem.
		\item Je to ferromagnetický kov s Curieovou teplotou 1115~$^\circ$C.
		\item Vytváří dvě krystalové modifikace:
		\begin{enumerate}
			\item hcp -- nejtěsnější hexagonální uspořádání
			\item fcc -- kubická, plošně centrovaná mřížka
		\end{enumerate}
		\item Fázová změna hcp na fcc nastává okolo teploty 420~$^\circ$C.\footnote[frame]{\href{https://doi.org/10.1016/0001-6160(89)90007-2}{Study of the h.c.p.-f.c.c. phase transition in cobalt by acoustic measurements}}
		\item Má jediný stabilní izotop \ce{^{59}Co}.
		\item Známe 28 radioizotopů, nejdelší poločas rozpadu má \ce{^{60}Co}, 5,3 roku.
	\end{itemize}
	\vfill
}

\frame{
	\frametitle{}
	\vfill
	\begin{itemize}
		\item Kobalt je méně reaktivní než železo.
		\item Na vzduchu se pokrývá tenkou vrstvou oxidu, která jej chrání před další oxidací.
		\item Zahříváním v přítomnosti kyslíku poskytuje \ce{Co3O4}, který se za vyšší teploty rozkládá na \ce{CoO}.
		\item Reakcí s halogeny poskytuje binární halogenidy.
		\item Nereaguje s vodíkem, ani dusíkem, a to ani při zahřívání.
		\item Za vyšší teploty reaguje s borem, uhlíkem, fosforem, arsenem a sírou.
		\item Nejvyšší oxidační číslo je IV, ale to je poměrně vzácné.
		\item Běžné jsou sloučeniny v oxidačních stavech II a III.
		\item Oxidační stav I a nižší se vyskytuje hlavně v organokovových sloučeninách.
	\end{itemize}
	\vfill
}

\subsection{Nikl}
\frame{
	\frametitle{}
	\vfill
	\begin{itemize}
		\item Nikl je magnetický za laboratorní teploty, Curieova teplota je 354~$^\circ$C.
		\item Má dvě možné elektronové konfigurace, které jsou si energeticky velice blízké:
		\begin{itemize}
			\item 3d$^8$ 4s$^2$
			\item 3d$^9$ 4s$^1$
		\end{itemize}
		\item Krystaluje v kubické plošně centrované mřížce.
		\item Má dobrou kujnost a tažnost, snadno se zpracovává.
		\item Má pět stabilních izotopů:
		\item Známe 26 radioizotopů, nejdelší poločas rozpadu má \ce{^{59}Ni}, 76 000 let.
	\end{itemize}
	\begin{tabular}{|l|r@{,}l|}
		\hline
		\ce{^{58}Ni} & 68 & 01 \% \\\hline
		\ce{^{60}Ni} & 26 & 22 \% \\\hline
		\ce{^{61}Ni} & 1 & 14 \% \\\hline
		\ce{^{62}Ni} & 3 & 64 \% \\\hline
		\ce{^{64}Ni} & 0 & 93 \% \\\hline
	\end{tabular}
	\vfill
}

\frame{
	\frametitle{}
	\vfill
	\begin{itemize}
		\item \textit{Nikl} se při zahřívání na vzduchu pokrývá vrstvou oxidu.
		\item V práškovém stavu je pyroforický.
		\item Za tepla se slučuje s borem, křemíkem, fosforem, sírou a halogeny.
		\item Oproti jiným kovům reaguje s fluorem pomalu.
		\item Je odolný vůči alkalickým hydroxidům, v minerálních kyselinách se pomalu rozpouští.
		\item Vytváří sloučeniny v oxidačních číslech $-$I až IV, nejběžnějším stavem je II.
		\item Ve sloučeninách dosahuje koordinačního čísla až 7.
	\end{itemize}
	\vfill
}

\section{Výskyt a získávání}
\subsection{Železo}
\frame{
	\frametitle{}
	\vfill
	\begin{itemize}
		\item Železo je čtvrtým nejrozšířenějším prvkem v zemské kůře (po kyslíku, křemíku a hliníku), jeho koncentrace je okolo 6~\%.\footnote[frame]{\href{https://www.mindat.org/element/Iron}{The mineralogy of Iron}}
		\item Je popsáno téměř 900 minerálů obsahujících železo, průmyslový význam mají čtyři:
		\begin{itemize}
			\item Hematit, \ce{Fe2O3}
			\item Magnetit, \ce{Fe3O4}
			\item Limonit, \ce{2Fe2O3.3H2O}
			\item Siderit, \ce{FeCO3}
		\end{itemize}
		\item V roce 2020 bylo celosvětově vytěženo téměř 2,5 miliardy tun železných rud.\footnote[frame]{\href{https://www.usgs.gov/centers/nmic/iron-ore-statistics-and-information}{Iron Ore Statistics and Information}}
		\item Největšími producenty jsou Austrálie, Brazílie, Čína a Indie.
	\end{itemize}
	\vfill
}

\frame{
	\frametitle{}
	\vfill
	\begin{itemize}
		\item Obrovská množství železa a niklu jsou uložena v Zemském jádře.
		\item Odhadovaná hmotnost zemského jádra je 2.10$^{24}$~kg, zhruba 80~\% tvoří železo.
		\item Struktura jádra není dosud přesně objasněna.
		\item Vnější jádro má teplotu 3 000--4 000 $^\circ$C.
		\item Vnitřní jádro má teplotu 5500 $^\circ$C.
	\end{itemize}
	\begin{figure}
		\adjincludegraphics[height=.42\textheight]{img/Earth_poster.png}
		\caption*{Vnitřní struktura Země.\footnote[frame]{Zdroj: \href{https://commons.wikimedia.org/wiki/File:Earth_poster.svg}{Kelvinsong/Commons}}}
	\end{figure}
	\vfill
}

\frame{
	\frametitle{}
	\vfill
	\textbf{Hematit}
	\begin{itemize}
		\item Trigonální minerál, \ce{Fe2O3}, červená až černá barva.\footnote[frame]{\href{https://mineraly.sci.muni.cz/oxidy/hematit.html}{Hematit}}
		\item Mezi běžné nečistoty patří titan, hliník a mangan.\footnote[frame]{\href{https://www.mindat.org/min-1856.html}{Hematite}}
		\item Byl detekován i na Marsu.\footnote[frame]{\href{https://doi.org/10.1029/2003JE002233}{Formation of the hematite‐bearing unit in Meridiani Planum: Evidence for deposition in standing water}}
	\end{itemize}
	\begin{columns}
		\begin{column}{.5\textwidth}
			\begin{figure}
				\adjincludegraphics[height=.32\textheight]{img/Andradite-Hematite-cktsr-36a.jpg}
				\caption*{Hematit a andradit, Severní Kapsko.\footnote[frame]{Zdroj: \href{https://commons.wikimedia.org/wiki/File:Andradite-Hematite-cktsr-36a.jpg}{Robert M. Lavinsky/Commons}}}
			\end{figure}
		\end{column}
		\begin{column}{.5\textwidth}
			\begin{figure}
				\adjincludegraphics[height=.32\textheight]{img/Calcite-Hematite-26306.jpg}
				\caption*{Hematit a kalcit, Namibie.\footnote[frame]{Zdroj: \href{https://commons.wikimedia.org/wiki/File:Calcite-Hematite-26306.jpg}{Robert M. Lavinsky/Commons}}}
			\end{figure}
		\end{column}
	\end{columns}
	\vfill
}

\frame{
	\frametitle{}
	\vfill
	\textbf{Limonit}
	\begin{itemize}
		\item Amorfní minerál, \ce{2Fe2O3.3H2O}, žlutá až hnědá barva.\footnote[frame]{\href{http://web.natur.cuni.cz/ugmnz/mineral/mineral/limonit.html}{Limonit}}
		\item Vzniká zvětráváním minerálů železa a srážením z vod.\footnote[frame]{\href{https://www.mindat.org/min-2402.html}{Limonite}}
	\end{itemize}
	\begin{figure}
		\adjincludegraphics[height=.4\textheight]{img/Limonite.jpg}
		\caption*{Limonit a magnetit, USA.\footnote[frame]{Zdroj: \href{https://commons.wikimedia.org/wiki/File:BIF_ventifact_(limonite-magnetite_banded_iron_formation,_Archean;_Ferris_Dune_Field,_Windy_Gap,_Wyoming,_USA)_1_(26052651216).jpg}{James St. John/Commons}}}
	\end{figure}
	\vfill
}

\frame{
	\frametitle{}
	\vfill
	\textbf{Siderit}
	\begin{itemize}
		\item Trigonální minerál, \ce{FeCO3}, žlutá až černá barva, může být i bezbarvý.\footnote[frame]{\href{https://mineraly.sci.muni.cz/karbonaty/siderit.html}{Siderit}}
		\item Patří mezi biominerály, protože je produkován bakteriemi (oxidací elementárního železa).\footnote[frame]{\href{https://www.mindat.org/min-3647.html}{Siderite}}
	\end{itemize}
	\begin{figure}
		\adjincludegraphics[height=.4\textheight]{img/Siderite.jpg}
		\caption*{Limonit a magnetit, USA.\footnote[frame]{Zdroj: \href{https://en.wikipedia.org/wiki/File:Harvard_Museum_of_Natural_History._Siderite._Gilman,_Eagle_Co.,_CO_(DerHexer)_2012-07-20.jpg}{DerHexer/Commons}}}
	\end{figure}
	\vfill
}

\frame{
	\frametitle{}
	\begin{columns}
		\begin{column}{.7\textwidth}
			\vfill
			\textbf{Magnetit}
			\begin{itemize}
				\item Kubický minerál, \ce{Fe3O4}, černá barva.\footnote[frame]{\href{https://mineraly.sci.muni.cz/oxidy/magnetit.html}{Magnetit}}
				\item Má strukturu spinelu, je magnetický.\footnote[frame]{\href{https://www.mindat.org/min-2538.html}{Magnetite}}
				\item Krystalky magnetitu jsou součástí některých živých organismů, jako např. magnetocitlivých bakterií, včel, holubů aj. Pravděpodobně jim slouží k orientaci podle magnetického pole Země.
				\item Dříve se využíval k výrobě magnetofonových pásek a nahrávacích hlav.\footnote[frame]{\href{https://mineralmilling.com/magnetite-recording-colourants-fischer-tropsch-process/}{Magnetite: Uses and Applications in Recording Media, Pigments/Dyes and the Fischer-Tropsch Process, Water Purification and Soil Remediation}}
			\end{itemize}
			\vfill
		\end{column}
		\begin{column}{.35\textwidth}
			\begin{figure}
				\adjincludegraphics[width=\textwidth]{img/Magnetite-163979.jpg}
				\caption*{Magnetit, USA.\footnote[frame]{Zdroj: \href{https://commons.wikimedia.org/wiki/File:Magnetite-163979.jpg}{Robert M. Lavinsky/Commons}}}
			\end{figure}
		\end{column}
	\end{columns}
}

\frame{
	\frametitle{}
	\vfill
	\begin{itemize}
		\item Čisté železo se vyrábí jen v malé míře, buď redukcí čistého oxidu nebo termickým rozkladem pentakarbonylu železa.
		\item Pentakarbonyl železa, \ce{[Fe(CO)5]}, je oranžová až žlutá kapalina.
		\item Vyrábí se reakcí čistého železa s oxidem uhelnatým za tlaku až 30~MPa a teploty 150--200~$^\circ$C.
		\item Při teplotě 250~$^\circ$C se rozkládá na práškové železo, které se označuje jako \textit{karbonylové železo}.
		\item \ce{Fe + 5 CO ->[150 $^\circ$C, 30 MPa] Fe(CO)5}
		\item \ce{Fe(CO)5 ->[250 $^\circ$C] Fe + 5 (CO)}
	\end{itemize}
	\vfill
}

\frame{
	\frametitle{}
	\vfill
	\begin{columns}
		\begin{column}{.5\textwidth}
			\begin{figure}
				\adjincludegraphics[width=\textwidth]{img/Fe_CO_5.png}
			\end{figure}
		\end{column}
		\begin{column}{.5\textwidth}
			\begin{figure}
				\adjincludegraphics[height=.7\textheight]{img/Sample_of_iron_pentacarbonyl.jpg}
				\caption*{Pentakarbonylželeza.\footnote[frame]{Zdroj: \href{https://commons.wikimedia.org/wiki/File:Sample_of_iron_pentacarbonyl.jpg}{Smokefoot/Commons}}}
			\end{figure}
		\end{column}
	\end{columns}
	\vfill
}

\frame{
	\frametitle{}
	\vfill
	\begin{itemize}
		\item \textit{Oceli} jsou slitiny železa s uhlíkem a dalšími prvky, které obsahují \textit{méně než 2,14~\% uhlíku}.
		\item Při vyšším obsahu uhlíku se slitiny označují jako \textit{litiny}.
		\item Vyrábí se v ocelárnách ze surového železa nebo železného šrotu snižováním obsahu uhlíku a dalších prvků (S, P, N) a přidáváním vhodných legujících prvků.
		\item Surové železo se získává ve vysokých pecích redukcí železné rudy (\ce{Fe2O3.FeO}) koksem.
	\end{itemize}

	\begin{columns}
		\begin{column}{.5\textwidth}
			\begin{figure}
				\adjincludegraphics[width=0.85\textwidth]{img/SteelMill_interior.jpg}
				\caption*{Ocelárna.\footnote[frame]{Zdroj: \href{https://commons.wikimedia.org/wiki/File:SteelMill_interior.jpg}{Payton Chung/Commons}}}
			\end{figure}
		\end{column}

		\begin{column}{.5\textwidth}
			\begin{figure}
				\adjincludegraphics[width=0.85\textwidth]{img/The_viaduct_La_Polvorilla,_Salta_Argentina.jpg}
				\caption*{Ocelový most v Argentině.\footnote[frame]{Zdroj: \href{https://commons.wikimedia.org/wiki/File:The_viaduct_La_Polvorilla,_Salta_Argentina.jpg}{Alicia Nijdam/Commons}}}
			\end{figure}
		\end{column}
	\end{columns}
	\vfill
}

\frame{
	\frametitle{}
	\vfill
	\begin{figure}
		\adjincludegraphics[width=.88\textwidth]{img/Iron_carbon_phase_diagram.png}
		\caption*{Fázový diagram železo-uhlík.\footnote[frame]{Zdroj: \href{https://commons.wikimedia.org/wiki/File:Iron_carbon_phase_diagram.svg}{AG Caesar/Commons}}}
	\end{figure}
	\vfill
}

\frame{
	\frametitle{}
	\vfill
	\begin{itemize}
		\item Surovinou jsou železné rudy:
		\begin{itemize}
			\item Magnetit -- \ce{Fe3O4}
			\item Hematit -- \ce{Fe2O3}
			\item Limonit -- \ce{Fe2O3 . n H2O}
			\item Siderit -- \ce{FeCO3}
		\end{itemize}
		\item Výroba oceli probíhá ve \textit{vysoké peci}, což je 30--50 m vysoká pec, vyzděná žáruvzdorným materiálem. Teplota uvnitř pece může dosahovat až 2000~$^\circ$C.\footnote[frame]{\href{https://www.youtube.com/watch?v=b3BOMfH7Dbc}{Jak se vyrábí železo a ocel}}
		\item Výroba oceli je kontinuální proces, který může běžet bez výhasu i několik let.
		\item Shora se pec zaváží směsí železné rudy, koksu a vápence (struskotvorné látky\footnote[frame]{\href{http://www.strusky.cz/}{Struska}}). Zdola se do pece vhání předehřátý vzduch s vyšší koncentrací kyslíku.
		\item Koks reaguje s kyslíkem za uvolnění tepla a vzniku CO, který následně redukuje železnou rudu.
		\begin{itemize}
			\item \ce{2 C + O2 -> 2 CO}
			\item \ce{CO2 + C -> 2 CO}
		\end{itemize}
	\end{itemize}
	\vfill
}

\frame{
	\frametitle{}
	\vfill
	\begin{itemize}
		\item Železná ruda se zde redukuje koksem a písky a jíly se převádí pomocí vápence na strusku.
		\item Ruda a koks se v peci temperují a postupně se snižuje podíl kyslíku:
		\item \ce{Fe2O3 -> Fe3O4 -> FeO -> Fe}
		\item Maximální teplota je ve spodní části pece, ze které se získává tavenina oceli. Ta se odlévá do ingotů, příp. do jiných tvarů.
		\item V současnosti se vyrábí ocel s obsahem uhlíku v rozmezí 0,5--1,5~\% a malým množstvím síry a fosforu.
	\end{itemize}
	\begin{figure}
		\adjincludegraphics[height=0.32\textheight]{img/Chains_of_pig_iron.jpg}
		\caption*{Surové železo.\footnote[frame]{Zdroj: \href{https://commons.wikimedia.org/wiki/File:Chains_of_pig_iron_casting_machine_07.jpg}{Blast furnace chip worker/Commons}}}
	\end{figure}
	\vfill
}

\frame{
	\frametitle{}
	\vfill
	\begin{itemize}
		\item Kromě železa získáváme z vysoké pece ještě \textit{vysokopecní plyn} a \textit{strusku}.
		\item \textbf{Vysokopecní plyn} obsahuje hořlavé látky, ale má poměrně malou výhřevnost. Využívá se hlavně k temperaci vzduchu, který vstupuje do vysoké pece. Obsahuje hlavně CO, \ce{H2}, \ce{CH4}, \ce{CO2} a \ce{N2}.
		\item \textbf{Struska} obsahuje hlavně \ce{SiO2}, \ce{Al2O3}, \ce{CaO}, má podobu černých granulí.
		\item Využívá se jako přísada do cementů.
	\end{itemize}
	\begin{figure}
		\adjincludegraphics[height=0.33\textheight]{img/Struska.jpg}
		\caption*{Struska.\footnote[frame]{Zdroj: \href{https://commons.wikimedia.org/wiki/File:Rauašlakk.jpg}{Minnekon/Commons}}}
	\end{figure}
	\vfill
}

\frame{
	\frametitle{}
	\begin{figure}
		\adjincludegraphics[height=0.7\textheight]{img/Blast_furnace_NT.png}
		\caption*{Schéma vysoké pece.\footnote[frame]{Zdroj: \href{https://commons.wikimedia.org/wiki/File:Blast_furnace_NT.PNG}{Tosaka/Commons}}}
	\end{figure}
	\vfill
}

\frame{
	\frametitle{}
	\begin{figure}
		\adjincludegraphics[height=0.7\textheight]{img/Haut_fourneau.jpg}
		\caption*{Horní část vysoké pece.\footnote[frame]{Zdroj: \href{https://commons.wikimedia.org/wiki/File:Haut_fourneau_P6_Gueulard_Enfer_2.jpg}{Borvan53/Commons}}}
	\end{figure}
	\vfill
}

\frame{
	\frametitle{}
	\begin{figure}
		\adjincludegraphics[height=0.7\textheight]{img/Trinec-zelezarny.jpg}
		\caption*{Třinecké železárny.\footnote[frame]{Zdroj: \href{https://commons.wikimedia.org/wiki/File:Třinecké_železárny.jpg}{Ondřej Žváček/Commons}}}
	\end{figure}
	\vfill
}

\frame{
	\frametitle{}
	\begin{figure}
		\adjincludegraphics[height=0.7\textheight]{img/Trinecke-zelezarny-vysoka-pec.jpg}
		\caption*{Vysoká pec v Třineckých železárnách.\footnote[frame]{Zdroj: \href{https://commons.wikimedia.org/wiki/File:Trinecke-zelezarny-vysoka-pec-c6-539.jpg}{Viktor Mácha/Commons}}}
	\end{figure}
	\vfill
}

\subsection{Kobalt}
\frame{
	\frametitle{}
	\vfill
	\begin{itemize}
		\item Koncentrace v zemské kůře je 20--30~ppm.
		\item Je přítomen i v meteorickém železe.
		\item Známe téměř 60 minerálů obsahujících kobalt.\footnote[frame]{\href{https://www.mindat.org/element/Cobalt}{The mineralogy of Cobalt}}
		\item Nejdůležitější jsou:
		\begin{enumerate}
			\item Kobaltin, \ce{CoAsS}
			\item Safflorit, \ce{CoAs2}
			\item Glaukodot, \ce{(Co,Fe)AsS}
			\item Skutterudit, \ce{CoAs3}
		\end{enumerate}
		\item Malá množství jsou přítomna i v cigaretovém kouři.\footnote[frame]{\href{https://doi.org/10.3390/ijerph8020613}{Hazardous Compounds in Tobacco Smoke}}
	\end{itemize}
	\vfill
}

\frame{
	\frametitle{}
	\vfill
	\textbf{Kobaltin}
	\begin{itemize}
		\item Orthorombický minerál, \ce{CoAsS}, stříbřitá až šedá barva, často narůžovělý.\footnote[frame]{\href{https://mineraly.sci.muni.cz/sulfidy/kobaltin.html}{Kobaltin}}
		\item Důležitý zdroj kobaltu.\footnote[frame]{\href{https://www.mindat.org/min-1093.html}{Cobaltite}}
	\end{itemize}
	\begin{columns}
		\begin{column}{.5\textwidth}
			\begin{figure}
				\adjincludegraphics[height=.4\textheight]{img/Cobaltite-d05-67a.jpg}
				\caption*{Kobaltin, Chile.\footnote[frame]{Zdroj: \href{https://commons.wikimedia.org/wiki/File:Erythrite-Cobaltite-202101.jpg}{Robert M. Lavinsky/Commons}}}
			\end{figure}
		\end{column}
		\begin{column}{.5\textwidth}
			\begin{figure}
				\adjincludegraphics[height=.4\textheight]{img/Erythrite-Cobaltite-202101.jpg}
				\caption*{Kobaltin, Švédsko.\footnote[frame]{Zdroj: \href{https://commons.wikimedia.org/wiki/File:Cobaltite-d05-67a.jpg}{Robert M. Lavinsky/Commons}}}
			\end{figure}
		\end{column}
	\end{columns}
	\vfill
}

\frame{
	\frametitle{}
	\begin{columns}
		\begin{column}{.7\textwidth}
			\vfill
			\textbf{Safflorit}
			\begin{itemize}
				\item Orthorombický minerál, \ce{(Co,Ni,Fe)As2}, cínově bílá až šedá barva.\footnote[frame]{\href{https://mineraly.sci.muni.cz/sulfidy/safflorit.html}{Safflorit}}
				\item Kromě kobaltu může obsahovat nikl a železo.\footnote[frame]{\href{https://www.mindat.org/min-3500.html}{Safflorite}}
				\item Pokud obsahuje více než 50~\% železa, jde o \textit{loellingit}.
				\item Pokud obsahuje více než 50~\% niklu, jde o \textit{rammelsbergit}.
			\end{itemize}
			\vfill
		\end{column}
		\begin{column}{.3\textwidth}
			\begin{figure}
				\adjincludegraphics[height=.4\textheight]{img/Safflorite2.jpg}
				\caption*{Safflorit, Německo.\footnote[frame]{Zdroj: \href{https://commons.wikimedia.org/wiki/File:Safflorite-Lollingite-Rammelsbergite-55551.jpg}{Robert M. Lavinsky/Commons}}}
			\end{figure}
		\end{column}
	\end{columns}
}

\frame{
	\frametitle{}
	\vfill
	\textbf{Glaukodot}
	\begin{itemize}
		\item Orthorombický minerál, \ce{(Co,Fe)AsS}, cínově bílá barva.\footnote[frame]{\href{https://www.mindat.org/min-3500.html}{Safflorite}}
		\item Typicky je poměr Co:Fe roven 3:1, může obsahovat i malý podíl niklu.
	\end{itemize}
	\begin{columns}
		\begin{column}{.5\textwidth}
			\begin{figure}
				\adjincludegraphics[height=.4\textheight]{img/Glaucodot-md47c.jpg}
				\caption*{Glaukodot, Švédsko.\footnote[frame]{Zdroj: \href{https://commons.wikimedia.org/wiki/File:Glaucodot-md47c.jpg}{Robert M. Lavinsky/Commons}}}
			\end{figure}
		\end{column}
		\begin{column}{.5\textwidth}
			\begin{figure}
				\adjincludegraphics[height=.4\textheight]{img/Glaucodot-rare08-56b.jpg}
				\caption*{Glaukodot, Švédsko.\footnote[frame]{Zdroj: \href{https://commons.wikimedia.org/wiki/File:Glaucodot-rare08-56b.jpg}{Robert M. Lavinsky/Commons}}}
			\end{figure}
		\end{column}
	\end{columns}
	\vfill
}

\frame{
	\frametitle{}
	\vfill
	\textbf{Skutterudit}
	\begin{itemize}
		\item Kubický minerál, \ce{(Co,Ni)As3}, cínově bílá až šedá barva.\footnote[frame]{\href{https://mineraly.sci.muni.cz/sulfidy/skutterudit.html}{Skutterudit}}
		\item Těží se jako ruda kobaltu, niklu a arsenu.\footnote[frame]{\href{https://www.mindat.org/min-3500.html}{Safflorite}}
	\end{itemize}
	\begin{figure}
		\adjincludegraphics[height=.4\textheight]{img/Skutterudite.jpg}
		\caption*{Skutterudit, Maroko.\footnote[frame]{Zdroj: \href{https://commons.wikimedia.org/wiki/File:Skutterudite_(vein_in_a_fault_system_in_the_Bou_Azzer-El_Graara_Ophiolite;_Tamdrost_Mine,_Bou_Azzer_Mining_District,_Morocco)_6.jpg}{James St. John/Commons}}}
	\end{figure}
	\vfill
}

\frame{
	\frametitle{}
	\vfill
	\begin{itemize}
		\item Kobalt se získává jako vedlejší produkt při výrobě niklu, mědi a olova.
		\item Způsob přípravy závisí na kovu, který jej v rudě doprovází.
		\item Rudy se praží, čímž se odstraní struska a získá se směs kovů a oxidů.
		\item Extrakcí kyselinou sírovou se převedou do roztoku železo, kobalt a nikl. Měď zůstává nerozpuštěna.
		\item Kobalt se oxiduje chlornanem a sráží se jako hydroxid:
		\item \ce{2 Co^{2+} + ClO- + 4 OH- + H2O -> 2 Co(OH)3 + Cl-}
		\item Vzniklý hydroxid se termicky dehydratuje a získaný oxid se redukuje dřevěným uhlím nebo aluminotermicky.
	\end{itemize}
	\vfill
}

\subsection{Nikl}
\frame{
	\frametitle{}
	\vfill
	\begin{itemize}
		\item Koncentrace v zemské kůře je okolo 90~ppm, je sedmým nejrozšířenějším přechodným kovem.
		\item V přírodě jej nacházíme v ryzím stavu i ve formě rud.\footnote[frame]{\href{https://www.mindat.org/element/Nickel}{The mineralogy of Nickel}}
		\item Nachází se i v meteoritech, kde jeho obsah dosahuje až 10~\%.
		\item Velké množství niklu je pravděpodobně uloženo v zemském jádře.
		\item Nejrozšířenější minerálu niklu jsou:
		\begin{enumerate}
			\item Pentlandit, \ce{(NiFe)9S8}
			\item Millerit, \ce{NiS}
			\item Gersdorffit, \ce{NiAsS}
		\end{enumerate}
		\item Kovový nikl se těží např. v Sudbury v Kanadě.\footnote[frame]{\href{https://uwaterloo.ca/earth-sciences-museum/resources/mining-canada/mining-history-sudbury-area}{The mining history of the Sudbury area}}
	\end{itemize}
	\vfill
}

\frame{
	\frametitle{}
	\begin{columns}
		\begin{column}{.7\textwidth}
			\vfill
			\textbf{Pentlandit}
			\begin{itemize}
				\item Kubický minerál, \ce{(NiFe)9S8}, bronzově žlutá barva.\footnote[frame]{\href{http://mineraly.sci.muni.cz/sulfidy/pentlandit.html}{Pentlandit}}
				\item Zpravidla obsahuje více niklu než železa.\footnote[frame]{\href{https://www.mindat.org/min-3155.html}{Pentlandite}}
				\item Nejdůležitější ruda niklu.
				\item Byl studován jako katalyzátor pro elektrolytickou výrobu vodíku.\footnote[frame]{\href{https://doi.org/10.1038/ncomms12269}{Pentlandite rocks as sustainable and stable efficient electrocatalysts for hydrogen generation}}
			\end{itemize}
			\vfill
		\end{column}
		\begin{column}{.3\textwidth}
			\begin{figure}
				\adjincludegraphics[height=.35\textheight]{img/Pentlandite.jpg}
				\caption*{Petlandit, Norsko.\footnote[frame]{Zdroj: \href{https://commons.wikimedia.org/wiki/File:Pentlandite,_Pyrrhotite-540342.jpg}{John Sobolewski/Commons}}}
			\end{figure}
		\end{column}
	\end{columns}
}

\frame{
	\frametitle{}
	\vfill
	\textbf{Millerit}
	\begin{itemize}
		\item Trigonální minerál, \ce{NiS}, bronzově žlutá až nazelenalá barva.\footnote[frame]{\href{http://mineraly.sci.muni.cz/sulfidy/millerit.html}{Millerit}}
		\item Je elektricky vodivý.
		\item Naleziště jsou v Karlovarském kraji (Jáchymově).\footnote[frame]{\href{https://www.mindat.org/min-2711.html}{Millerite}}
	\end{itemize}
	\begin{columns}
		\begin{column}{.5\textwidth}
			\begin{figure}
				\adjincludegraphics[height=.38\textheight]{img/Millerite-44389.jpg}
				\caption*{Millerit, Austrálie.\footnote[frame]{Zdroj: \href{https://commons.wikimedia.org/wiki/File:Millerite-44389.jpg}{Robert M. Lavinsky/Commons}}}
			\end{figure}
		\end{column}
		\begin{column}{.5\textwidth}
			\begin{figure}
				\adjincludegraphics[height=.38\textheight]{img/Calcite-Millerite-d06-24b.jpg}
				\caption*{Kalcit a millerit, USA.\footnote[frame]{Zdroj: \href{https://commons.wikimedia.org/wiki/File:Calcite-Millerite-d06-24b.jpg}{Robert M. Lavinsky/Commons}}}
			\end{figure}
		\end{column}
	\end{columns}
	\vfill
}

\frame{
	\frametitle{}
	\vfill
	\textbf{Gersdorffit}
	\begin{itemize}
		\item Kubický minerál, \ce{NiAsS}, ocelově šedá barva.\footnote[frame]{\href{http://mineraly.sci.muni.cz/sulfidy/gersdorffit.html}{Gersdorffit}}
		\item Naleziště jsou i v Karlovarském kraji (Jáchymově).\footnote[frame]{\href{https://www.mindat.org/min-1683.html}{Gersdorffite}}
		\item Existuje ve třech krystalových formách.\footnote[frame]{\href{https://rruff-2.geo.arizona.edu/uploads/AM67_1058.pdf}{A further crystal structure refinement of gersdorffite}}
	\end{itemize}
	\begin{figure}
		\adjincludegraphics[height=.4\textheight]{img/Gersdorffit.jpg}
		\caption*{Gersdorffit, Německo.\footnote[frame]{Zdroj: \href{https://commons.wikimedia.org/wiki/File:Gersdorffit_-_Lobenstein,_Vogtland.jpg}{Ra'ike/Commons}}}
	\end{figure}
	\vfill
}

\frame{
	\frametitle{}
	\vfill
	\textbf{Abelsonit}
	\begin{itemize}
		\item Triklinický minerál, \ce{C31H32N4Ni}, cínově bílá až šedá barva.\footnote[frame]{\href{https://www.mindat.org/min-1.html}{Abelsonite}}
		\item Velmi vzácný minerál, jediný známý krystalický geoporfyrin.
	\end{itemize}
	\begin{columns}
		\begin{column}{.5\textwidth}
			\begin{figure}
				\adjincludegraphics[height=.5\textheight]{img/Abelsonit.png}
			\end{figure}
		\end{column}
		\begin{column}{.5\textwidth}
			\begin{figure}
				\adjincludegraphics[height=.4\textheight]{img/Abelsonite.jpg}
				\caption*{Abelsonit, USA.\footnote[frame]{Zdroj: \href{https://commons.wikimedia.org/wiki/File:Abelsonite_-_Green_River_Formation,_Uintah_County,_Utah,_USA.jpg}{Thomas Witzke/Commons}}}
			\end{figure}
		\end{column}
	\end{columns}
	\vfill
}

\frame{
	\frametitle{}
	\vfill
	\begin{itemize}
		\item Hlavním zdrojem pro výrobu niklu jsou sulfidické rudy, které je možné lépe zpracovat než oxidické rudy (flotace, magnetická separace).
		\item V případě rud obsahujících měď a železo se nejprve koncentrát Ni/Cu kalcinuje s křemenem, čímž se odstraní velká část sulfidů a železa.
		\item Vzniklá tavenina se nechá pomalu chladnout, tím dojde k separaci sulfidů a slitiny Ni/Cu.
		\item Získaný \ce{Ni3S2} se pražením převede na oxid, který se buď využije přímo k přípravě ocelí nebo se redukuje na nikl.
		\item V případě elektrolytické rafinace niklu se oxid redukuje uhlíkem, z takto získaného niklu se vyrobí anoda.
		\item Elektrolytická rafinace se provádí ve vodném roztoku \ce{NiSO4} nebo \ce{NiCl2}, jako katoda slouží čistý nikl.
		\item Takto připravený nikl má čistotu 99,9~\%.
		\item Druhou možností rafinace je \textit{Mondův proces}.\footnote[frame]{\href{https://www.oxfordreference.com/display/10.1093/oi/authority.20110803100205384}{Mond process}}
	\end{itemize}
	\vfill
}

\frame{
	\frametitle{}
	\vfill
	\begin{itemize}
		\item Mondův, nebo karbonylový, proces je založen na tvorbě těkavého karbonylu, který je následně rozložen.
		\item Tento proces byl vyvinut v roce 1890 Ludwigem Mondem.\footnote[frame]{\href{https://doi.org/10.1039/CT8905700749}{Action of carbon monoxide on nickel}}$^,$\footnote[frame]{\href{https://doi.org/10.1038/059063a0}{The Extraction of Nickel from its Ores by the Mond Process}}
		\item Redukce se provádí vodním plynem, což je směs vodíku a oxidu uhelnatého. Získává se vedením vodní páry přes rozžhavený koks:
		\ce{C + H2O -> CO + H2}
		\item Vodík redukuje oxid na kov:
		\item \ce{NiO + H2 ->[200 $^\circ$C] Ni + H2O}
		\item Surový nikl reaguje s oxidem uhelnatým za vzniku plynného tetrakarbonylu:
		\item \ce{Ni + 4 CO ->[60 $^\circ$C] Ni(CO)4}
		\item Ten se pak termicky rozkládá, čistota připraveného niklu je 99,95~\%:
		\item \ce{Ni(CO)4 ->[200 $^\circ$C] Ni + 4 CO}
	\end{itemize}
	\vfill
}

\frame{
	\frametitle{}
	\vfill
	\begin{columns}
		\begin{column}{.5\textwidth}
			\begin{figure}
				\adjincludegraphics[height=.55\textheight]{img/Nickel_kugeln.jpg}
				\caption*{Nikl připravený Mondovým proces.\footnote[frame]{Zdroj: \href{https://commons.wikimedia.org/wiki/File:Nickel_kugeln.jpg}{René Rausch/Commons}}}
			\end{figure}
		\end{column}
		\begin{column}{.5\textwidth}
			\begin{figure}
				\adjincludegraphics[height=.55\textheight]{img/Nickel_electrolytic_and_1cm3_cube.jpg}
				\caption*{Elektrolytický nikl.\footnote[frame]{Zdroj: \href{https://commons.wikimedia.org/wiki/File:Nickel_electrolytic_and_1cm3_cube.jpg}{Alchemist-hp/Commons}}}
			\end{figure}
		\end{column}
	\end{columns}
	\vfill
}

\section{Využití}
\subsection{Železo}
\frame{
	\frametitle{}
	\vfill
	\begin{columns}
		\begin{column}{.7\textwidth}
			\begin{itemize}
				\item Železo je nejpoužívanější kov.
				\item V roce 2019 byly vyrobeny téměř dvě miliardy tun oceli, nejvíce v Číně a Indii.\footnote[frame]{\href{https://www.worldsteel.org/steel-by-topic/statistics/steel-data-viewer_new/P1_crude_steel_total_pub/CHN/IND}{Total production of crude steel}}
				\item V Česku bylo vyrobeno v roce 2019 téměř 4,5 miliónu tun oceli (polovina v Třinci).\footnote[frame]{\href{https://www.ocelarskaunie.cz/globalni-produkce-surove-oceli-za-rok-2019-se-oproti-roku-2018-zvysila-o-34/}{Globální produkce surové oceli za rok 2019 se oproti roku 2018 zvýšila o 3,4 \%}}
				\item S ocelí se setkáváme prakticky ve všech oblastech:
				\begin{itemize}
					\item Stavebnictví, betonářství
					\item Automobilový a dopravní průmysl
					\item Stavební technika
				\end{itemize}
			\end{itemize}
		\end{column}

		\begin{column}{.4\textwidth}
			\begin{figure}
				\adjincludegraphics[width=\textwidth]{img/Rebar_01.jpg}
				\caption*{Betonářská ocel.\footnote[frame]{Zdroj: \href{https://commons.wikimedia.org/wiki/File:Rebar_01.JPG}{Vsolymossy/Commons}}}
			\end{figure}
		\end{column}
	\end{columns}
	\vfill
}

\frame{
	\frametitle{}
	\vfill
	\begin{figure}
		\adjincludegraphics[width=\textwidth]{img/UKB-Stavba-SIMU-2019-2.jpg}
		\caption*{Litý železobeton, stavba SIMU+}
	\end{figure}
	\vfill
}

\frame{
	\frametitle{}
	\vfill
	\begin{itemize}
		\item \textbf{Legovaná ocel} vzniká přidáváním dalších prvků do nízkouhlíkové oceli.\footnote[frame]{\href{http://svanda.webz.cz/vyuka/legury.htm}{Vliv legovacích prvků na vlastnosti ocelí}} Cílem je optimalizace vlastností.
		\begin{itemize}
			\item K legování se využívají hlavně mangan, molybden, nikl, chrom, vanad a křemík.
			\item Podle obsahu legujících prvků rozlišujeme oceli nízkolegované (do 4~\% jiných kovů), středně legované (5–10~\%) a vysokolegované (nad 10~\%).
		\end{itemize}
		\item \textbf{Kalení} je proces, kdy se ocel vyžíhá a prudce zchladí. Tím dojde ke vzniku nerovnovážných, martenzitických, struktur. Zakalením se zvýší pevnost v tahu a dojde k nárůstu objemu.\footnote[frame]{\href{http://www.tumlikovo.cz/zakladni-kaleni/}{Základní kalení}}
		\begin{itemize}
			\item Kalení se provádí zpravidla ve vodě nebo oleji. Některé oceli lze kalit i vzduchem, příp. tlakovým vzduchem.
			\item Ve vodě dochází k velmi rychlému poklesu teploty, komplikací je vznik tzv. \textit{parního polštáře}, ten se potlačuje pohybem kaleného předmětu.
		\end{itemize}
	\end{itemize}
	\vfill
}

\frame{
	\frametitle{}
	\vfill
	\begin{itemize}
		\item \textbf{Popouštění} se provádí zpravidla po zakalení, ocel se zahřeje na dostatečně vysokou teplotu, aby se odstranilo vnitřní pnutí vzniklé zakalením, ale nesmí dojít k fázovým změnám. Tímto snížíme tvrdost materiálu a zvýšíme jeho houževnatost.
	\end{itemize}
	\begin{figure}
		\adjincludegraphics[height=.45\textheight]{img/Tempering_standards.jpg}
		\caption*{Vzorky oceli popouštěné při různých teplotách, barva materiálu odpovídá teplotě popouštění.\footnote[frame]{Zdroj: \href{https://commons.wikimedia.org/wiki/File:Tempering_standards_used_in_blacksmithing.JPG}{Zaereth/Commons}}}
	\end{figure}
	\vfill
}

\frame{
	\frametitle{}
	\vfill
	\begin{itemize}
		\item Konstrukční ocel -- zpravidla nelegovaná ocel využívaná ve stavebnictví a strojírenství.
		\item Betonářská ocel -- nelegovaná, příp. nízko legovaná ocel využívaná pro armování betonu.
		\item Elektrotechnická ocel -- minimální obsah uhlíku, 1--4,5~\% křemíku. Používá se pro výrobu plechů pro jádra transformátorů.
		\item Jeřábová ocel -- nízký obsah uhlíku (do 0,5~\%), legovaná Cr, Ni, Mo, V, Ti a Nb.
		\begin{itemize}
			\item V letech 2007--2008 způsobil nedostatek jeřábové oceli problémy při výstavbě nových mrakodrapů.\footnote[frame]{\href{https://www.nbcnews.com/id/wbna18909319}{Building boom causing a shortage in cranes}}
		\end{itemize}
		\item Nerezová ocel -- vysoký obsah chromu (12--30~\%), niklu (až 30~\%) nebo manganu (až 24~\%). Odolná vůči korozi, využívá se v automobilovém, chemickém a potravinářském průmyslu, ve stavebnictví a jinde.
	\end{itemize}
	\vfill
}

\frame{
	\frametitle{}
	\vfill
	\begin{itemize}
		\item Nástrojová ocel -- středně až vysoce legovaná ocel. Vrtáky, nože na kovy, frézy.
		\item Damascénská (damašková) ocel -- skupina ocelí určená pro výrobu mečů, šavlí a dalších chladných zbraní. Vyznačují se vysokou pružností a pevností.\footnote[frame]{\href{https://www.chladnezbrane.eu/clanky-a-recenze/vyroba-a-puvod-damaskove-oceli-strucne-damaskova-ocel-damascus-steel/}{Damašková ocel}}
		\item Ocel se vyrábí postupným spojováním jednotlivých vrstev.
		\item Spojují se oceli s nízkým obsahem uhlíku s vysokouhlíkovými. Analýzami bylo zjištěno, že při výrobě dochází i ke vzniku uhlíkových nanotrubic.\footnote[frame]{\href{https://www.nationalgeographic.com/science/article/carbon-nanotechnology-in-an-17th-century-damascus-sword}{Carbon nanotechnology in an 17th century Damascus sword}}
	\end{itemize}
	\begin{figure}
		\adjincludegraphics[width=.65\textwidth]{img/DamaszenerKlinge.jpg}
		\caption*{Nůž z damascénské oceli.\footnote[frame]{Zdroj: \href{https://commons.wikimedia.org/wiki/File:DamaszenerKlinge.JPG}{Ralf Pfeifer/Commons}}}
	\end{figure}
	\vfill
}

\frame{
	\frametitle{}
	\vfill
	\begin{columns}
		\begin{column}{.55\textwidth}
			\begin{itemize}
				\item Alnico -- skupina železných slitin obsahujících železo, hliník, nikl a kobalt. Dále mohou obsahovat měď a titan.
				\item Jsou ferromagnetické a využívají pro výrobu velmi silných magnetů. Silnější jsou jen neodymové magnety.
				\item Jejich Currieova teplota je okolo 800~$^\circ$C, ale používají se jen do teploty 600~$^\circ$C.\footnote[frame]{\href{https://www.magnety.cz/alnico/}{ALNICO}} Jsou vysoce odolné vůči kyselinám a rozpouštědlům.
				\item Vyrábějí se slinováním nebo sléváním.
			\end{itemize}
		\end{column}
		\begin{column}{.55\textwidth}
			\begin{figure}
				\adjincludegraphics[width=.95\textwidth]{img/AlNiCo.jpg}
				\caption*{AlNiCo magnet.\footnote[frame]{Zdroj: \href{https://commons.wikimedia.org/wiki/File:Magnetron_magnet.JPG}{Chetvorno/Commons}}}
			\end{figure}
		\end{column}
	\end{columns}
	\vfill
}

\frame{
	\frametitle{}
	\begin{columns}
		\begin{column}{.7\textwidth}
			\vfill
			\begin{itemize}
				\item \textbf{Železobeton} je kompozitní materiál skládající se z betonu a oceli.
				\item První použití bylo zaznamenáno již v druhé polovině 19. století.
				\item Důvodem pro využití oceli v betonových konstrukcích je nízká pevnost v tahu betonu. Beton je odolný vůči namáhání tlakem, ale při tahovém a smykovém namáhání je odolnost výrazně nižší.
				\item To lze kompenzovat vyztužením betonu ocelovými tyčemi, zpravidla s kruhovým průřezem.
			\end{itemize}
			\vfill
		\end{column}
		\begin{column}{.35\textwidth}
			\begin{figure}
				\adjincludegraphics[width=\textwidth]{img/SagradaFamiliaRoof2.jpg}
				\caption*{Stavba železobetonové střechy Sagrada Familia.\footnote[frame]{Zdroj: \href{https://commons.wikimedia.org/wiki/File:SagradaFamiliaRoof2.jpg}{Etan J. Tal/Commons}}}
			\end{figure}
		\end{column}
	\end{columns}
}

\frame{
	\frametitle{}
	\vfill
	\begin{figure}
		\adjincludegraphics[width=\textwidth]{img/UKB-Stavba-SIMU-2019-2.jpg}
		\caption*{Litý železobeton, stavba SIMU+}
	\end{figure}
	\vfill
}


\frame{
	\frametitle{}
	\textbf{Baterie na bázi \ce{Fe^{VI}}}
	\begin{itemize}
		\item Alternativa klasických suchých článků, založená na železanech (\textit{super-iron}).
		\item Místo \ce{MnO2} obsahují \ce{K2FeO4}:\footnote[frame]{\href{https://doi.org/10.1126/science.285.5430.1039}{Energetic Iron(VI) Chemistry: The Super-Iron Battery}}
		\item \ce{2 FeO$_4^{2-}$ + 5 H2O + 6 e- <=> Fe2O3 + 10 OH-}
		\item Celkovou reakci lze zapsat:
		\item \ce{2 K2FeO4 + 3 Zn -> Fe2O3 + ZnO + 2 K2ZnO2}
		\item Výhodou je možnost nabíjení baterií.
		\item Možnou oblastí využití jsou baterie pro kardiostimulátory, ale i pro elektromobily, kde by mohlo železo nahradit dražší a hůře dostupné lithium.\footnote[frame]{\href{https://doi.org/10.3390/en3050960}{A High Capacity Li-Ion Cathode: The Fe(III/VI) Super-Iron Cathode}}
	\end{itemize}
}

\subsection{Kobalt}
\frame{
	\frametitle{}
	\begin{itemize}
		\item Velká část kobaltu se používá pro výrobu slitin.
		\item Jejich vysoká teplotní stabilita umožňuje konstrukci tepelně namáhaných částí plynových turbín.\footnote[frame]{\href{https://pubs.usgs.gov/fs/2011/3081/pdf/fs2011-3081.pdf}{Cobalt—For Strength and Color}}
		\item Také se využívají, díky korozní odolnosti, k výrobě ortopedických implantátů.
	\end{itemize}
	\begin{figure}
		\adjincludegraphics[height=.4\textheight]{img/Jet_engine.png}
		\caption*{Proudový motor.\footnote[frame]{Zdroj: \href{https://commons.wikimedia.org/wiki/File:Jet_engine.svg}{Jeff Dahl/Commons}}}
	\end{figure}
}

\frame{
	\frametitle{}
	\begin{itemize}
		\item V Li-ION bateriích se využívá jako katoda \ce{LiCoO2}, formální oxidační číslo kobaltu je +III.
		\item Vyrábí se zahříváním \ce{Li2CO3} s \ce{Co3O4} a následnou několikahodinovou kalcinací v kyslíkové atmosféře.
	\end{itemize}
	\begin{figure}
		\adjincludegraphics[height=.5\textheight]{img/Lithium-cobalt-oxide-3D-balls.png}
		\caption*{Krystalová struktura \ce{LiCoO2}.\footnote[frame]{Zdroj: \href{https://commons.wikimedia.org/wiki/File:Lithium-cobalt-oxide-3D-balls.png}{Ben Mills/Commons}}}
	\end{figure}
}

\frame{
	\frametitle{}
	\begin{itemize}
		\item Během nabíjení dochází k oxidaci části kobaltu na oxidační stav IV, vzniká nestechiometrická fáze \ce{Li_xCoO2}.
		\item Tyto typy baterií mají stabilní kapacitu, ale dosahují nižších hodnot proudu i nábojové hustoty.
		\item Při zahřátí na vyšší teploty nebo přebíjení může dojít k explozi, protože se uvolňuje kyslík, který může reagovat s organickým elektrolytem baterie.
	\end{itemize}
	\begin{figure}
		\adjincludegraphics[height=.35\textheight]{img/Li-Ion_batteries_for_mobile_phones.jpg}
		\caption*{Li-Ion akumulátory.\footnote[frame]{Zdroj: \href{https://commons.wikimedia.org/wiki/File:Li-Ion_batteries_for_mobile_phones.jpg}{Phrontis/Commons}}}
	\end{figure}
}

\frame{
	\frametitle{}
	\begin{columns}
		\begin{column}{.65\textwidth}
		\begin{itemize}
		\item Radioizotop \ce{^{60}Co} se využívá jako zdroj $\gamma$ záření v medicínských aplikacích.
		\item Má poločas rozpadu 5,27 let.
		\item Vyrábí se v jaderném reaktoru ostřelováním kovového kobaltu nebo slitiny kobaltu a niklu neutrony v reaktoru:
		\item \ce{^{59}Co + n -> ^{60}Co + \gamma}
		\item Pro radioterapii nádorů se využívá slitina wolframu, do které je uložen kobaltový zářič.
		\item Radioizotop \ce{^{57}Co} se využívá jako zdroj $\gamma$ záření pro Mösbauerovu spektroskopii železa a jeho sloučenin.
	\end{itemize}
	\end{column}
	\begin{column}{.4\textwidth}
	\begin{figure}
		\adjincludegraphics[width=.9\textwidth]{img/Cobalt-60_Irradiator.jpg}
		\caption*{Pasterace potravin.\footnote[frame]{Zdroj: \href{https://commons.wikimedia.org/wiki/File:Cobalt-60_Irradiator.tif}{US Department of Energy/Commons}}}
	\end{figure}
	\end{column}
	\end{columns}
}

\frame{
	\frametitle{}
	\begin{columns}
		\begin{column}{.7\textwidth}
			\begin{itemize}
				\item Sloučeniny kobaltu se využívají jako katalyzátory ve Fischer--Tropschových syntézách.\footnote[frame]{\href{https://doi.org/10.1016/j.cjche.2017.11.001}{Cobalt catalysts for Fischer–Tropsch synthesis: The effect of support, precipitant and pH value}}
				\item Jde o reakce, které slouží k syntéze uhlovodíků.
				\item Kobalt se využívá hlavně v případě, kdy je výchozím zdrojem zemní plyn.
				\item \ce{CH4 + H2O -> CO + 3 H2}
				\item \ce{(2n + 1) H2 + n CO -> C_nH_{2n+2} + n H2O}
				\item Katalyzátor je immobilizován na povrchu siliky, aluminy nebo zeolitu.
			\end{itemize}
		\end{column}
		\begin{column}{.35\textwidth}
			\begin{figure}
				\adjincludegraphics[width=\textwidth]{img/HCCo3(CO)9.png}
			\end{figure}
		\end{column}
	\end{columns}
}

\subsection{Nikl}
\frame{
	\frametitle{}
	\begin{columns}
		\begin{column}{.7\textwidth}
			\vfill
			\begin{itemize}
				\item Většina niklu se využívá pro výrobu nerezové oceli.
				\begin{itemize}
					\item 68~\% nerezová ocel
					\item 10~\% neželezné slitiny
					\item 9~\% elektropokovování
					\item 7~\% legování oceli
					\item 3~\% slévárenství
					\item 4~\% jiné využití (NiCd, NiMH články, atd.)
				\end{itemize}
				\item Obsah niklu nad 8~\% usnadňuje tvorbu austenitu\footnote[frame]{Tuhý roztok uhlíku v $\gamma$-železe} a tím zvyšuje odolnost vůči korozi a zlepšuje zpracovatelnost.\footnote[frame]{\href{https://www.worldstainless.org/Files/issf/non-image-files/PDF/TheStainlessSteelFamily.pdf}{The Stainless Steel Family}}
			\end{itemize}
			\vfill
		\end{column}
		\begin{column}{.35\textwidth}
			\begin{figure}
				\adjincludegraphics[width=\textwidth]{img/Sink_and_taps.jpg}
				\caption*{Nerezová ocel.\footnote[frame]{Zdroj: \href{https://commons.wikimedia.org/wiki/File:Sink_and_taps_in_the_men\%27s_locker_room_2_-_BW.jpg}{W.carter/Commons}}}
			\end{figure}
		\end{column}
	\end{columns}
}

\frame{
	\frametitle{}
	\begin{columns}
	\begin{column}{.7\textwidth}
			\vfill
			\textbf{Monel}
			\begin{itemize}
				\item Neželezná slitina: 68~\% Ni, 32~\% Cu a stopová množství Mn, Si, C a Fe.
				\item Slitina byla poprvé vyrobena v roce 1901 R. C. Stanleyem. Byla pojmenována po prezidentovi společnosti INCO A. Monellovi.
				\item Velmi obtížně se zpracovává, ale je odolná vůči mechanickému i chemickému namáhání v širokém rozmezí teplot.\footnote[frame]{\href{https://www.worldstainless.org/Files/issf/non-image-files/PDF/TheStainlessSteelFamily.pdf}{The Stainless Steel Family}}
				\item Využívá se např. pro konstrukci aparatur pro práci s plynným fluorem.
			\end{itemize}
		\vfill
	\end{column}
	\begin{column}{.35\textwidth}
		\begin{figure}
			\adjincludegraphics[height=.3\textheight]{img/Monel-valves.jpg}
			\caption*{Ventil z monelu.\footnote[frame]{Zdroj: \href{https://commons.wikimedia.org/wiki/File:Monel-valves--The-Alloy-Valve-Stockist.JPG}{Heather Smith/Commons}}}
		\end{figure}
	\end{column}
	\end{columns}
}

\frame{
	\frametitle{}
	\begin{columns}
		\begin{column}{.6\textwidth}
			\begin{itemize}
				\item Galvanické niklování zvyšuje odolnost materiálu vůči korozi.
				\item Alternativou je chemické niklování, které nevyžaduje vodivé povrchy.
				\item Vrstva niklu a fosforu vzniká redoxní reakcí.
				\item Ta by měla být autokatalytická, aby se zabránilo vylučování niklu v niklovací lázni.
				\item Zpravidla se využívá síran nikelnatý a vhodné redukční činidlo, např. fosforitan nebo borohydrid:
			\end{itemize}
		\end{column}
		\begin{column}{.4\textwidth}
			\begin{figure}
				\adjincludegraphics[width=.80\textwidth]{img/Copper_electroplating.png}
				\caption*{Galvanické pokovování.\footnote[frame]{Zdroj: \href{https://commons.wikimedia.org/wiki/File:Copper_electroplating_principle_(multilingual).svg}{Torsten Henning/Commons}}}
			\end{figure}
		\end{column}
	\end{columns}
	\ce{2 Ni^{2+} + 8 H2PO2- + 2 H2O -> 2 Ni + 6 H2PO3 + 2 H+ + 2 P + 3 H2}
}

\frame{
	\frametitle{}
	\textbf{Raneyův nikl}
	\begin{itemize}
		\item Šedý prášek, složený převážně z niklu, vyrábí se ze slitiny niklu s hliníkem.
		\item Ta je rozpuštěna v hydroxidu sodném, hliník se rozpustí za vývoje vodíku, který se nasorbuje na povrch zrn niklu.\footnote[frame]{\href{https://www.masterorganicchemistry.com/2011/09/30/reagent-friday-raney-nickel/}{Reagent Friday: Raney Nickel}}
		\item \ce{2 Al + 2 NaOH + 6 H2O -> 2 Na[Al(OH)4] + 3 H2}
		\item Díky obsahu vodíku je Raneyův nikl pyroforický a musí se uchovávat tak, aby se zabránilo kontaktu se vzdušným kyslíkem.\footnote[frame]{\href{https://www.youtube.com/watch?v=KEOcufv1qdM}{Raney Nickel spontaneous combustion}}
	\end{itemize}
	\begin{figure}
		\adjincludegraphics[height=.26\textheight]{img/Dry_Raney_nickel.jpg}
		\caption*{Suchý Raneyův nikl.\footnote[frame]{Zdroj: \href{https://commons.wikimedia.org/wiki/File:Dry_Raney_nickel.jpg}{Rune.welsh/Commons}}}
	\end{figure}
}

\frame{
	\frametitle{}
	\begin{itemize}
		\item Využívá se jako hydrogenační katalyzátor v organické syntéze. Poprvé byl použit v roce 1926 americkým inženýrem Murrayem Raneyem k hydrogenaci rostlinného oleje.
		\item V současnosti se používá ve velkém množství průmyslových aplikací, např. při redukci benzenu na cyklohexan.\footnote[frame]{\href{https://doi.org/10.1016/0166-9834(83)80051-7}{The nature of raney nickel, its adsorbed hydrogen and its catalytic activity for hydrogenation reactions (review)}}
	\end{itemize}
	\begin{figure}
		\adjincludegraphics[height=.35\textheight]{img/Hydrogenace.png}
	\end{figure}
}

\frame{
	\frametitle{}
	\begin{itemize}
		\item Dále ho lze využít k desulfurizaci.
	\end{itemize}
	\begin{figure}
		\adjincludegraphics[width=\textwidth]{img/Thiofene-reduction.png}
	\end{figure}
	\begin{itemize}
	\item Nebo redukci funkčních skupin, např. nitro na amino.
	\end{itemize}
	\begin{figure}
		\adjincludegraphics[width=\textwidth]{img/Nitro-reduction.png}
	\end{figure}
}

\section{Sloučeniny}
\subsection{Železo}
\frame{
	\frametitle{}
	\begin{columns}
		\begin{column}{.6\textwidth}
			\begin{tabular}{|l|l|}
		\hline
		\textbf{Oxidační stav} & \textbf{Příklad} \\\hline
		$-$2 & \ce{Na2[Fe(CO)4]} \\\hline
		$-$1 & \ce{Fe2(CO)8} \\\hline
		0 & \ce{Fe(CO)5} \\\hline
		1 & \ce{[($\eta$_5-C5H5)Fe(CO)2]2} \\\hline
		2 & ferrocen, síran železnatý \\\hline
		3 & \ce{FeCl3}, \ce{[Fe(C5H5)2]BF4} \\\hline
		4 & \ce{Fe(diars)2Cl$_2^{2+}$} \\\hline
		5 & \ce{FeO$_4^{3-}$} \\\hline
		6 & \ce{FeO$_4^{2-}$} \\\hline
		7 & \ce{FeO$_4^-$}, jen v matrici\footnote[frame]{\href{https://doi.org/10.1039/C6CP06753K}{Experimental and theoretical identification of the Fe(VII) oxidation state in \ce{FeO$_4^-$}}} \\\hline
	\end{tabular}
	\end{column}
	\begin{column}{.4\textwidth}
		\begin{figure}
			\adjincludegraphics[width=\textwidth]{img/Fp2SingleNoFeFe.png}
		\end{figure}
	\end{column}
	\end{columns}
}

\frame{
	\frametitle{}
	\vfill
	\textbf{Oxidy železa}
	\begin{itemize}
		\item Známe mnoho oxidů v oxidačních číslech II a III. Nejběžnější jsou:
		\item \ce{FeO}
		\item \ce{Fe3O4 = FeO.Fe2O3}
		\item \ce{Fe3O4}
		\item \textbf{Oxid železnatý}, \ce{FeO}, můžeme připravit tepelným rozkladem šťavelanu železnatého v inertní atmosféře:
		\item \ce{Fe(COO)2 ->[Ar] FeO + CO2 + CO}
		\item Má strukturu NaCl, železnaté i oxidové ionty mají oktaedrickou koordinaci.
	\end{itemize}
	\vfill
}

\frame{
	\frametitle{}
	\begin{columns}
		\begin{column}{.7\textwidth}
			\vfill
			\textbf{Podvojné oxidy železa}
			\begin{itemize}
				\item Nejvýznamnější jsou granáty a ferity.
				\item Připravují se zahříváním oxidu železitého s příslušným uhličitanem.
				\item Mají spinelovou strukturu nebo inverzní.
				\item \textbf{Oxid železnato-železitý}, \ce{Fe3O4}, můžeme ho popsat jako \ce{Fe$^{\textrm{II}}$Fe$_2^{\textrm{III}}$O4}.
				\item Minerál \textit{magnetit}.
				\item Lze ho připravit tzv. Schikorrovou reakcí, tepelným rozkladem hydroxidu železnatého v anaerobním prostředí:\footnote[frame]{\href{https://dx.doi.org/10.1186/1556-276X-8-16}{Facile synthesis of ultrathin magnetic iron oxide nanoplates by Schikorr reaction}}
				\item \ce{3 Fe(OH)2 -> Fe3O4 + H2 + 2 H2O}
			\end{itemize}
			\vfill
		\end{column}
		\begin{column}{.35\textwidth}
			\begin{figure}
				\adjincludegraphics[width=\textwidth]{img/Fe3O4.jpg}
				\caption*{Oxid železnato-železitý.\footnote[frame]{Zdroj: \href{https://commons.wikimedia.org/wiki/File:Fe3O4.JPG}{Leiem/Commons}}}
			\end{figure}
		\end{column}
	\end{columns}
}

\frame{
	\frametitle{}
	\vfill
	\begin{figure}
		\adjincludegraphics[height=.7\textheight]{img/Kristallstruktur_Magnetit.png}
		\caption*{Struktura magnetitu, železnaté ionty jsou zelené, železité modré.\footnote[frame]{Zdroj: \href{https://commons.wikimedia.org/wiki/File:Kristallstruktur_Magnetit.png}{David Schrupp/Commons}}}
	\end{figure}
	\vfill
}

\frame{
	\frametitle{}
	\vfill
	\textbf{Koroze}
	\begin{itemize}
		\item Samovolná degradace kovů a jiných materiálů způsobená chemickou nebo elektrochemickou reakcí s látkami v okolním prostředí.
		\item Velký ekonomický a technologický problém.
		\item Náklady tvoří jak snaha zabránit korozi, tak náprava následků.
		\item Ve vyspělých zemích jde o 3 až 5~\% HDP.
	\end{itemize}
	\begin{figure}
		\adjincludegraphics[width=0.5\textwidth]{img/Rusty_car_in_River.jpg}
		\caption*{Zkorodované auto.\footnote[frame]{Zdroj: \href{https://commons.wikimedia.org/wiki/File:Rusty_car_in_River-03\%2B_(403730921).jpg}{Sheba/Commons}}}
	\end{figure}
	\vfill
}

\frame{
	\frametitle{}
	\vfill
	\begin{itemize}
		\item \textit{Chemická koroze} -- koroze způsobená chemickou reakcí v nevodivém prostředí.
		\item \textit{Elektrochemická koroze} -- koroze způsobená elektrochemickou reakcí v elektrolytech.
		\item \textit{Atmosférická koroze} -- nejběžnější druh koroze, interakce s kyslíkem, vlhkostí a dalšími plyny (\ce{SO2}, CO, \ce{NH3}). Podle korozní agresivity rozdělujeme atmosféry do šesti tříd (C1--C5, CX).
		\item \textit{Koroze v kapalinách} -- nejčastěji se jedná o předměty ponořené do vody, o rychlosti koroze rozhoduje převážně hodnota tvrdosti vody, pH a množství rozpuštěných plynů.
		\item \textit{Půdní koroze} -- půda se skládá z pevné, kapalné a plynné složky, na korozi má nejvyšší vliv kapalná složka, která uděluje půdě vodivost.
	\end{itemize}
	\vfill
}

\frame{
	\frametitle{}
	\vfill
	\textbf{Antikorozní ochrana}
	\begin{itemize}
		\item Volba vhodného materiálu -- s ohledem na parametry prostředí a~mechanické a chemické namáhání.
		\item Konstrukční řešení -- omezení vibrací, otěrů, tepelného namáhání, kontaktu s~agresivními látkami.
		\item Úprava prostředí
		\item Povrchová úprava -- modifikace povrchu materiálem odolným vůči korozi.
		\item Elektrochemická ochrana -- obětovaná anoda.
	\end{itemize}
	\textbf{Povrchová úprava}
	\begin{itemize}
		\item Galvanické pokovování -- ionty kovu z elektrolytu se vylučují na povrchu katody.
		\item Opačně zapojený galvanický článek.
		\item Nejčastěji se využívají povrchy z~mědi, niklu, chromu, zinku a~kadmia.
	\end{itemize}
	\vfill
}

\frame{
	\frametitle{}
	\begin{columns}
		\begin{column}{0.7\textwidth}
			\vfill
			\begin{itemize}
				\item Měděné povlaky slouží zejména jako mezivrstvy pro složitější systémy pokovení nebo jako vlastní dekorativní vrstva.
				\item Niklové povlaky se vyznačují nepropustností pro korozní činidla a lesklým vzhledem. Proto mají funkci ochrannou i ozdobnou.
				\item Chromové vrstvy mají velkou odolnost proti korozi za normálních i zvýšených teplot, velkou tvrdost
				a otěruvzdornost.
				\item Zinkové vrstvy fungující na principu anodového účinku dobře chrání ocel před atmosférickou korozí.
			\end{itemize}
			\vfill
		\end{column}
		\begin{column}{0.3\textwidth}
			\begin{figure}
				\adjincludegraphics[width=\textwidth]{img/Copper_electroplating_principle.png}
				\caption*{Galvanické pokovování.\footnote[frame]{Zdroj: \href{https://commons.wikimedia.org/wiki/File:Copper_electroplating_principle_(multilingual).svg}{Torsten Henning/Commons}}}
			\end{figure}
		\end{column}
	\end{columns}
}

\frame{
	\frametitle{}
	\textbf{Obětovaná anoda}
	\begin{itemize}
		\item Založeno na principu anodické polarizace.
		\item Chráněný předmět se vodivě spojí s elektrodou z méně ušlechtilého kovu.
		\item Elektroda postupně koroduje a tím zabraňuje korozi předmětu.
		\item Využívá se hliník, hořčík a zinek.
		\item Tento druh ochrany se používá u nádrží, lodí, podzemních kabelech, apod.
	\end{itemize}
	\begin{figure}
		\adjincludegraphics[width=0.4\textwidth]{img/Anode_sacrificielle_en_zinc.jpg}
		\caption*{Obětovaná anoda.\footnote[frame]{Zdroj: \href{https://commons.wikimedia.org/wiki/File:Anode_sacrificielle_en_zinc_(1).JPG}{Jean-Pierre Bazard/Commons}}}
	\end{figure}
}

\frame{
	\frametitle{}
	\vfill
	\textbf{Halogenidy železa}\\
	\begin{tabular}{|c|c|c|c|}
		\hline
		\textbf{Halogenid} & \textbf{Barva} & \textbf{T$_\text{t}$ [$^\circ$C]} & \textbf{T$_\text{v}$ [$^\circ$C]} \\\hline
		\ce{FeF3} & zelený & $>$1000 & - \\\hline
		\ce{FeCl3} & hnědý & 308 & 316, rozklad \\\hline
		\ce{FeOCl} & fialový & - & - \\\hline
		\ce{FeBr3} & hnědý & 200, rozklad & - \\\hline
		\ce{FeI3} & černý & - & - \\\hline
		\ce{FeF2} & bezbarvý & 970 & 1100 \\\hline
		\ce{FeCl2} & zelený & 677 & 1023 \\\hline
		\ce{FeBr2} & žlutohnědý & 684 & 934 \\\hline
		\ce{FeI2} & bílošedý & 587 & 827 \\\hline
	\end{tabular}

	\begin{itemize}
		\item Bezvodé halogenidy železité se připravují reakcí prvků za zvýšené teploty.
		\item U bromidu železitého nesmíme překročit teplotu 200~$^\circ$C, jinak vzniká \ce{FeBr2}.
	\end{itemize}
	\vfill
}

\frame{
	\frametitle{}
	\begin{columns}
		\begin{column}{.7\textwidth}
			\vfill
			\textbf{Chlorid železitý}
			\begin{itemize}
				\item V bezvodé podobě má strukturu \ce{BiI3} tvořenou oktaedry \ce{FeCl6} propojenými chloridovými můstky. Připravuje se spalováním železa v chloru:
				\item \ce{2 Fe + 3 Cl2 -> 2 FeCl3}
				\item Strukturu hexahydrátu můžeme popsat vzorcem\\ \textit{trans}-\ce{[Fe(H2O)4Cl2]Cl.2H2O}.
				\item  V plynném stavu má dimerní strukturu (\ce{Fe2Cl6}), podobně jako \ce{AlCl3}.
			\end{itemize}
			\vfill
		\end{column}
		\begin{column}{.35\textwidth}
			\begin{figure}
				\adjincludegraphics[width=\textwidth]{img/Iron-trichloride-sheet-3D-polyhedra.png}
				\caption*{Krystalová struktura \ce{FeCl3}.\footnote[frame]{Zdroj: \href{https://commons.wikimedia.org/wiki/File:Iron-trichloride-sheet-3D-polyhedra.png}{Benjah-bmm27/Commons}}}
			\end{figure}
		\end{column}
	\end{columns}
	\vfill
}

\frame{
	\frametitle{}
	\vfill
	\textbf{Chlorid železitý}\\
	\begin{itemize}
		\item Používá se jako vločkovací činidlo při čištění vody.
		\item Dokáže rozpouštět kovovou měď, čehož se využívá při domácí výrobě plošných spojů:\footnote[frame]{\href{https://blog.venca-x.cz/leptani-plosneho-spoje-v-domacich-podminkach/}{Leptání plošného spoje v domácích podmínkách}}
		\item \ce{FeCl3 + Cu -> FeCl2 + CuCl}
		\item \ce{FeCl3 + CuCl -> FeCl2 + CuCl2}
		\item Jde o Lewisovu kyselinu, dokáže katalyzovat Friedel-Craftsovy reakce:\footnote[frame]{\href{https://doi.org/10.1021/ed073p272}{Iron(III) Chloride as a Lewis Acid in the Friedel-Crafts Acylation Reaction}}
	\end{itemize}

	\begin{figure}
		\adjincludegraphics[width=\textwidth]{img/FeCl3-catalyst.png}
	\end{figure}
	\vfill
}

\frame{
	\frametitle{}
	\vfill
	\begin{figure}
		\adjincludegraphics[height=.65\textheight]{img/Etched_in_Ferric_Chloride_at_Home.jpg}
		\caption*{Leptání plošného spoje v roztoku \ce{FeCl3}.\footnote[frame]{Zdroj: \href{https://commons.wikimedia.org/wiki/File:Etched_in_Ferric_Chloride_at_Home.jpg}{Adam Greig/Commons}}}
	\end{figure}
	\vfill
}

\frame{
	\frametitle{}
	\vfill
	\textbf{Zelená skalice}
	\begin{itemize}
		\item Heptahydrát síranu železnatého, \ce{FeSO4.7 H2O}, zelená krystalická látka.
		\item V přírodě se vyskytuje jako minerál melanterit.\footnote[frame]{\href{http://mineraly.sci.muni.cz/sulfaty/melanterit.html}{Melanterit}}
		\item Připravuje se rozpouštěním železa ve zředěné kyselině sírové nebo oxidací sulfidu:
		\item \ce{Fe + H2SO4 -> FeSO4 + H2}
		\item \ce{2 FeS2 + 7 O2 + 2 H2O -> 2 FeSO4 + 2 H2SO4}
		\item Při delším stání dochází k uvolňování vody a postupné oxidaci.
		\item Roztoky postupně mění barvu na hnědou, to je způsobeno oxidací vzdušným kyslíkem.
	\end{itemize}
	\vfill
}

\frame{
	\frametitle{}
	\vfill
	\begin{columns}
		\begin{column}{.5\textwidth}
			\begin{figure}
				\adjincludegraphics[height=.45\textheight]{img/Iron-sulfate-heptahydrate-sample.jpg}
				\caption*{Vzorek zelené skalice.\footnote[frame]{Zdroj: \href{https://commons.wikimedia.org/wiki/File:Iron(II)-sulfate-heptahydrate-sample.jpg}{Benjah-bmm27/Commons}}}
			\end{figure}
		\end{column}
		\begin{column}{.4\textwidth}
			\begin{figure}
				\adjincludegraphics[height=.45\textheight]{img/Iron-sulfate-heptahydrate-3D.png}
				\caption*{Krystalová struktura zelené skalice.\footnote[frame]{Zdroj: \href{https://commons.wikimedia.org/wiki/File:Iron(II)-sulfate-heptahydrate-3D-balls.tiff}{Smokefoot/Commons}}}
			\end{figure}
		\end{column}
	\end{columns}
	\vfill
}

\frame{
	\frametitle{}
	\vfill
	\textbf{Ferrocen}
	\begin{itemize}
		\item Organokovová sloučenina železa, obsahuje železnatý kation obklopený dvěma cyklopentadienidovými anionty.
		\item Železo má v této sloučenině oxidační číslo +II.
		\item Je stabilní na vzduchu, sublimuje za nízké teploty.
		\item Poprvé byl (náhodou) připraven ve čtyřicátých letech reakcí plynného cyklopentadienu s železným potrubím.\footnote[frame]{\href{https://doi.org/10.1002/anie.201201598}{At Least 60 Years of Ferrocene}}
	\end{itemize}
	\begin{columns}
		\begin{column}{.5\textwidth}
			\begin{figure}
				\adjincludegraphics[height=.35\textheight]{img/Ferrocene.png}
			\end{figure}
		\end{column}
		\begin{column}{.5\textwidth}
			\begin{figure}
				\adjincludegraphics[height=.35\textheight]{img/Ferrocene_powdered.jpg}
				\caption*{Vzorek ferrocenu.\footnote[frame]{Zdroj: \href{https://commons.wikimedia.org/wiki/File:Photo_of_Ferrocene_(powdered).JPG}{TMaster/Commons}}}
			\end{figure}
		\end{column}
	\end{columns}
	\vfill
}

\frame{
	\frametitle{}
	\vfill
	\begin{itemize}
		\item Pentakarbonyl železa, \ce{[Fe(CO)5]}, je oranžová až žlutá kapalina.\footnote[frame]{\href{https://doi.org/10.1021/jacs.2c01469}{Structure and Spectroscopy of Iron Pentacarbonyl, \ce{Fe(CO)5}}}
		\item Vyrábí se reakcí čistého železa s oxidem uhelnatým za tlaku až 30~MPa a teploty 150--200~$^\circ$C.
		\item Železo má v této sloučenině oxidační číslo 0.
		\item Má strukturu trigonální bipyramidy.
		\item Pozorujeme u něj \textit{Berryho pseudorotaci}.\footnote[frame]{\href{https://doi.org/10.1021/ja00159a011}{Exchange of axial and equatorial carbonyl groups in pentacoordinate metal carbonyls in the solid state.}}
	\end{itemize}
	\begin{figure}
		\adjincludegraphics[width=.9\textwidth]{img/Iron-pentacarbonyl-Berry-mechanism.png}
		\caption*{Výměna ligandů Berryho pseudorotací u pentakarbonylu železa.\footnote[frame]{Zdroj: \href{https://commons.wikimedia.org/wiki/File:Iron-pentacarbonyl-Berry-mechanism.png}{Benjah-bmm27/Commons}}}
	\end{figure}
	\vfill
}

\frame{
	\frametitle{}
	\begin{columns}
		\begin{column}{0.7\textwidth}
			\vfill
			\begin{itemize}
				\item V oxidačním čísle VI tvoří železanový anion, \ce{FeO$_4^{2-}$}.
				\item Vytváří světle fialové roztoky.
				\item Má silné oxidační vlastnosti, je stabilní pouze v silně bazických roztocích.
				\item Železany můžeme připravit oxidací železitého iontu v bazickém prostředí:
				\item \ce{2 Fe(OH)3 + 3 NaOCl + 4 NaOH -> 2 Na2FeO4 + 3 NaCl + 5 H2O}
				\item Z roztoku ho lze vysrážet jako barnatou sůl.
				\item Železany jsou velmi silná oxidační činidla:
				\item \small \ce{4 K2FeO4 + 4 H2O -> 3 O2 + 2 Fe2O3 + 8 KOH}
			\end{itemize}
			\vfill
		\end{column}
		\begin{column}{0.3\textwidth}
			\begin{figure}
				\adjincludegraphics[width=.9\textwidth]{img/Potassiumferrate-VI-solution.png}
				\caption*{Roztok železanu draselného.\footnote[frame]{Zdroj: \href{https://commons.wikimedia.org/wiki/File:Potassiumferrate(VI)solution.png}{RandomExperiments/Commons}}}
			\end{figure}
		\end{column}
	\end{columns}
}

\subsection{Kobalt}
\frame{
	\frametitle{}
	\vfill
	\textbf{Kobalt}
	\begin{itemize}
		\item Je méně reaktivní než železo.
		\item Na vzduchu se pokrývá vrstvou oxidu, která jej chrání před další oxidací.
		\item Zahříváním v přítomnosti kyslíku poskytuje \ce{Co3O4}, který se za vyšší teploty rozkládá na \ce{CoO}.
		\item Reakcí s halogeny poskytuje binární halogenidy.
		\item Nereaguje s vodíkem ani dusíkem a to ani při zahřívání.
		\item Za vyšší teploty reaguje s borem, uhlíkem, fosforem, arsenem a sírou.
		\item Nejvyšší oxidační číslo je IV, ale to je poměrně vzácné.
		\item Běžné jsou sloučeniny v oxidačních stavech II a III.
		\item Oxidační stav I a nižší se vyskytuje hlavně v organokovových sloučeninách.
	\end{itemize}
	\vfill
}

\frame{
	\frametitle{}
	\begin{columns}
		\begin{column}{.6\textwidth}
			\begin{tabular}{|l|c|}
				\hline
				\textbf{Oxidační stav} & \textbf{Příklad} \\\hline
				$-$1 & \ce{[Co(CO)4]-} \\\hline
				0 & \makecell[c]{\ce{[Co(PMe3)4]}\\ \ce{[Co2(CO)8]}} \\\hline
				1 & \ce{[Co(NCMe)5]+} \\\hline
				2 & \makecell[c]{\ce{CoCl2}\\ \ce{[CoCl4]^{2-}}} \\\hline
				3 & \ce{[Co(NH3)6]^{3+}} \\\hline
				4 & \makecell[c]{\ce{[CoF6]^{2-}} \\ \ce{[Co(1-norbornyl)4]}} \\\hline
			\end{tabular}
		\end{column}
		\begin{column}{.4\textwidth}
			\begin{figure}
				\adjincludegraphics[width=\textwidth]{img/Co-PMe3.png}
				\adjincludegraphics[width=\textwidth]{img/Tetrakis_1-norbornyl-cobalt.png}
			\end{figure}
		\end{column}
	\end{columns}
}

\frame{
	\frametitle{}
	\vfill
	\begin{itemize}
		\item Kobalt vytváří tři oxidy.
		\item \textbf{Oxid kobaltnatý}, CoO, šedozelená pevná látka.
		\item Vzniká tepelným rozkladem dusičnanu kobaltnatého nebo zahříváním práškového kobaltu na vzduchu.
		\item Při teplotách pod 16 $^\circ$C je \textit{antiferromagnetický}.
		\item Je součástí kobaltové modři využívané v keramice.
		\item  \textbf{Oxid kobaltnato-kobaltitý}, \ce{Co3O4}, černá pevná látka.
		\item Struktura se skládá z tetraedrů \ce{Co^{II}O4}, oktaedrů \ce{Co^{III}O6}. Kyslíky jsou koordinovány tetraedricky.\footnote[frame]{\href{https://doi.org/10.1016/0022-3697(64)90156-8}{The magnetic structure of \ce{Co3O4}}}
		\item Podobně jako CoO se využívá k barvení keramiky a také v některých lithiových akumulátorech.
		\item \textbf{Oxid kobaltitý}, \ce{Co2O3}, je také pevná černá látka.
		\item Připravuje se oxidací kobaltnatých solí:
		\item \ce{2 CoSO4 + 4 NaOH + NaOCl -> Co2O3 + 2 Na2SO4 + NaCl}
	\end{itemize}
	\vfill
}

\frame{
	\frametitle{}
	\vfill
	\textbf{Halogenidy kobaltu}
	\begin{itemize}
		\item V oxidačním stavu III známe pouze fluorid kobaltitý.
		\item \ce{CoF3} se využívá jako fluorační činidlo, je korozivní a má oxidační vlastnosti.
		\item V oxidačním stavu II známe všechny binární halogenidy.
	\end{itemize}
	\begin{tabular}{|c|c|c|c|}
		\hline
		\textbf{Sloučenina} & \textbf{Barva} & T$_t$ [$^\circ$C] & T$_v$ [$^\circ$C] \\\hline
		\ce{CoF3} & světle hnědý & 927 & - \\\hline
		\ce{CoF2} & růžový & 1217 & 1400 \\\hline
		\ce{CoCl2} & modrý & 726 & 1049 \\\hline
		\ce{CoBr2} & zelený & 678 & - \\\hline
		\ce{CoI2} & modročerný & 570 & - \\\hline
	\end{tabular}
	\vfill
}

\frame{
	\frametitle{}
	\vfill
	\begin{itemize}
		\item Barva chloridu kobaltnatého závisí na stupni hydratace. Bezvodý je světle modrý a při hydrataci postupně přechází až na červený hexahydrát.
		\item Toho se využívá např. v silikagelu k indikaci vlhkosti.
	\end{itemize}
	\begin{columns}
		\begin{column}{.4\textwidth}
			\begin{figure}
				\adjincludegraphics[height=.45\textheight]{img/Cobaltous_chloride_anhydrous.jpg}
				\caption*{Bezvodý \ce{CoCl2}.\footnote[frame]{Zdroj: \href{https://commons.wikimedia.org/wiki/File:Cobaltous_chloride_anhydrous.jpg}{W. Oelen/Commons}}}
			\end{figure}
		\end{column}

		\begin{column}{.4\textwidth}
			\begin{figure}
				\adjincludegraphics[height=.45\textheight]{img/Cobaltous_chloride.jpg}
				\caption*{\ce{CoCl2.6H2O}.\footnote[frame]{Zdroj: \href{https://commons.wikimedia.org/wiki/File:Cobaltous_chloride.jpg}{W. Oelen/Commons}}}
			\end{figure}
		\end{column}
	\end{columns}
	\vfill
}

\frame{
	\frametitle{}
	\begin{columns}
		\begin{column}{.75\textwidth}
			\begin{itemize}
				\item Halogenidy kobaltu, resp. jejich solváty, stály u zrodu chemie koordinačních sloučenin.
				\item Jejich struktura byla dlouho neznámá, o~její objasnění se zasloužil švédský chemik \emph{Alfred Werner}.
				\item Studoval solváty chloridu kobaltitého s~amoniakem. Zjistil, že existuje řada sloučenin, s~rozdílnou barvou.
				\item Tyto sloučeniny také poskytují rozdílná množství AgCl při reakci s~\ce{AgNO3}.
			\end{itemize}
		\end{column}
		\begin{column}{.35\textwidth}
			\begin{figure}
				\adjincludegraphics[width=\textwidth]{img/Werner.jpg}
				\caption*{Alfred Werner.\footnote[frame]{Zdroj: \href{https://commons.wikimedia.org/wiki/File:(UAZ)_AB.1.1093_Werner.tif}{UZH Archives/Commons}}}
			\end{figure}
		\end{column}
	\end{columns}
}

\frame{
	\frametitle{}
	\begin{itemize}
		\item \ce{[Co(NH3)5Cl]Cl2 + 2 AgNO3 -> 2 AgCl v + Co(NH3)5Cl](NO3)2}
	\end{itemize}
	\begin{tabular}{|l|l|l|l|}
		\hline
		\textbf{Sloučenina} & \textbf{Barva} & \textbf{Molů AgCl} & \textbf{Komplexní sloučenina} \\\hline
		\ce{CoCl3.4NH3} & Fialová & 1 & \ce{cis-[Co(NH3)4Cl2]Cl} \\\hline
		\ce{CoCl3.4NH3} & Zelená & 1 & \ce{trans-[Co(NH3)4Cl2]Cl} \\\hline
		\ce{CoCl3.5NH3} & Purpurová & 2 & \ce{[Co(NH3)5Cl]Cl2} \\\hline
		\ce{CoCl3.6NH3} & Žlutá & 3 & \ce{[Co(NH3)6]Cl3} \\\hline
	\end{tabular}

	\begin{figure}
		\adjincludegraphics[width=\textwidth]{img/Co-cistrans.png}
	\end{figure}
}

\subsection{Nikl}
\frame{
	\frametitle{}
	\vfill
	\begin{itemize}
		\item Při zahřívání na vzduchu se pokrývá vrstvou oxidu.
		\item V práškovém stavu je pyroforický.
		\item Za tepla se slučuje s borem, křemíkem, fosforem, sírou a halogeny.
		\item Oproti jiným kovům reaguje s fluorem pomalu.
		\item Je odolný vůči alkalickým hydroxidům, v minerálních kyselinách se pomalu rozpouští.
		\item Vytváří sloučeniny v oxidačních číslech $-$I až IV, nejběžnějším stavem je II.
		\item Ve sloučeninách dosahuje až koordinačního čísla 7.
	\end{itemize}
	\vfill
}

\frame{
	\frametitle{}
	\begin{columns}
		\begin{column}{.55\textwidth}
			\begin{tabular}{|l|c|}
				\hline
				\textbf{Ox. stav} & \textbf{Příklad} \\\hline
				$-$1 & \ce{[Ni2(CO)6]^{2-}} \\\hline
				0 & \makecell[c]{\ce{[Ni\{P(OC6H4-2-Me)3\}]}\\ \ce{[Ni(CO)4]}} \\\hline
				1 & \ce{[NiBr(PPh3)2]} \\\hline
				2 & \makecell[c]{\ce{[NiCl4]^{2-}}\\ \ce{[Ni(CN)5]^{3-}}\\ \ce{[Ni(H2O)6]^{2+}}\\ \ce{[Ni(H2O)2(dapbh)]^{2+}} (k.č. 7)} \\\hline
				3 & \makecell[c]{\ce{NiBr3(PEt3)2}\\ \ce{[NiF6]^{3-}}} \\\hline
				4 & \ce{[NiF6]^{2-}} \\\hline
			\end{tabular}
		\end{column}
		\begin{column}{.45\textwidth}
			\begin{figure}
				\adjincludegraphics[width=\textwidth]{img/Ni-dapbh.png}
			\end{figure}
		\end{column}
	\end{columns}
}

\frame{
	\frametitle{}
	\begin{columns}
		\begin{column}{.65\textwidth}
			\vfill
			\begin{itemize}
				\item \textbf{Oxid nikelnatý}, NiO, je zelená pevná látka.
				\item Krystaluje v kubické soustavě, strukturní typ NaCl.
				\item Lze jej připravit zahříváním niklu na vzduchu, ale takto vzniká často nestechiometrický.
				\item Nejjednodušší metodou je pyrolýza nikelnatých sloučenin, např. hydroxidu nebo halogenidů.
				\item Využívá se při výrobě solárních článku, NiCd a NiMH akumulátorů, apod.
			\end{itemize}
			\vfill
		\end{column}

		\begin{column}{.35\textwidth}
			\begin{figure}
				\adjincludegraphics[width=1.5\textwidth,rotate=90]{img/NiO.png}
				\caption*{Oxid nikelnatý.\footnote[frame]{Zdroj: \href{https://commons.wikimedia.org/wiki/File:Oxid_nikelnatý.PNG}{Ondřej Mangl/Commons}}}
			\end{figure}
		\end{column}
	\end{columns}
}

\section{Biologie}
\subsection{Železo}
\frame{
	\frametitle{}
	\begin{columns}
	\begin{column}{.7\textwidth}
		\vfill
		\begin{itemize}
			\item Železo je asi nejdůležitějším přechodným kovem pro biologii živočichů i rostlin.
			\item Tělo dospělého člověka obsahuje zhruba 4 g železa, z toho tři gramy připadají na \textit{hemoglobin}.
			\item Hemoglobin je bílkovina transportující kyslík, najdeme ho v červených krvinkách.\footnote[frame]{\href{https://www.wikiskripta.eu/w/Transport_kyslíku_krví}{Transport kyslíku krví}}
			\item Obsahuje železnatý ion ve vyskospinovém stavu komplexovaný porfyrinovým ligandem.
			\item Po navázání kyslíku, nedojde k oxidaci na \ce{Fe^{III}}, ale ke změně stavu na nízkospinový, diamagnetický. Zároveň se na železo váže histidin.
			\item Kromě kyslíku, transportuje hemoglobin i \ce{CO2}.
		\end{itemize}
		\vfill
	\end{column}
	\begin{column}{.35\textwidth}
		\begin{figure}
			\adjincludegraphics[width=\textwidth]{img/Oxygenated_vs_deoxygenated_RBC.jpg}
			\caption*{Okysličené a neokysličené červené krvinky.\footnote[frame]{Zdroj: \href{https://commons.wikimedia.org/wiki/File:Oxygenated_vs_deoxygenated_RBC.jpg}{Rogeriopfm/Commons}}}
		\end{figure}
	\end{column}
\end{columns}
}

\frame{
	\frametitle{}
	\begin{figure}
		\adjincludegraphics[width=1.1\textwidth]{img/Structures_of_Hemoglobin.png}
		\caption*{Hemoglobin s navázaným kyslíkem, oxidem uhličitým a oxidem uhelnatým.\footnote[frame]{Zdroj: \href{https://commons.wikimedia.org/wiki/File:Structures_of_Hemoglobin_forms.png}{Gladissk/Commons}}}
	\end{figure}
}

\frame{
	\frametitle{}
	\vfill
	\begin{itemize}
		\item Železo je součástí i jiných bílkovin, ty často obsahují vazbu \ce{Fe-S} (tzv. FeS proteiny).
		\item Železo je vázáno k postranním řetězcům aminokyselin \textit{cysteinu} a \textit{histidinu}.\footnote[frame]{\href{https://doi.org/10.1016/B978-0-12-378630-2.00222-X}{Iron–Sulfur Proteins}}
		\item Tyto proteiny mají funkci transferu elektronů (oxidoreduktasy nebo transelektronasy).
		\item Během transferu elektronů dochází ke změně oxidačního stavu železa z II na III, oba stavy jsou ve vysokospinové konfiguraci.
	\end{itemize}
	\begin{figure}
		\adjincludegraphics[height=.4\textheight]{img/Cystein-histidin.png}
	\end{figure}
	\vfill
}

\subsection{Kobalt}
\frame{
	\frametitle{}
	\vfill
	\begin{itemize}
		\item Kobalt je esenciální pro metabolismus všech živočichů.
		\item Je složkou vitamínu B12, označovaného jako \textit{kobalamin}.
		\item Vitamín byl objeven roku 1926 G. R. Minotem a W. P. Murphym.
		\item Jeho hlavní funkcí je regulace syntézy DNA, ale podílí se také na syntéze mastných kyselin a produkci energie.
		\item Bakterie v žaludku přežvýkavců dokáží zpracovat soli kobaltu na vitamín B12, proto je jeho přítomnost v půdě (v nízké koncentraci) důležitá pro zdraví pasoucích se zvířat.
		\item Na konci 19. století bylo zjištěno, že zhoubné onemocnění ovcí a hovězího dobytka je způsobeno právě nedostatkem kobaltu a nikoliv železa, jak se dříve předpokládalo.\footnote[frame]{\href{https://doi.org/10.1152/physrev.1952.32.1.66}{Cobalt, Copper and Molybdenum in the Nutrition of Animals and Plants}}
		\item U člověka způsobuje nedostatek vitamínu B12 chudokrevnost, únavu, zácpu, pokles váhy. Může způsobovat i neurologické změny (deprese).
	\end{itemize}
	\vfill
}

\frame{
	\frametitle{}
	\vfill
	\begin{figure}
		\adjincludegraphics[height=.67\textheight]{img/Cobalamin.png}
		\caption*{Struktura kobalaminu.\footnote[frame]{Zdroj: \href{https://commons.wikimedia.org/wiki/File:Cobalamin_skeletal.svg}{Hbf878/Commons}}}
	\end{figure}
	\vfill
}

\frame{
	\frametitle{}
	\vfill
	\begin{columns}
		\begin{column}{.6\textwidth}
			\begin{itemize}
				\item Hlavním zdrojem vitamínu B12 jsou živočišné produkty: maso, vejce, sýry.
				\item Doporučená denní dávka je 2--3~$\mu$g denně.
				\item Kobalamin je oranžová, diamagnetická látka.
				\item Koordinační sféra je obdobná, jako u železa v hemu.
				\item Kobalt je koordinován ke čtyřem dusíkům v rovině korrinového kruhu, pátý dusík je nad rovinou kruhu.
				\item Šestá pozice je obsazena uhlíkovým atomem z ligandu R.
			\end{itemize}
		\end{column}
		\begin{column}{.4\textwidth}
			\begin{figure}
				\adjincludegraphics[height=.63\textheight]{img/B12_1000mcg.jpg}
				\caption*{Vialka s vitamínem B12.\footnote[frame]{Zdroj: \href{https://commons.wikimedia.org/wiki/File:B12_1000mcg_vial_white_background.jpg}{Wesalius/Commons}}}
			\end{figure}
		\end{column}
	\end{columns}
	\vfill
}

\subsection{Nikl}
\frame{
	\frametitle{}
	\vfill
	\begin{itemize}
		\item Oproti železu a kobaltu je biologický význam niklu výrazně nižší.
		\item {[NiFe]} hydrogenáza je enzym katalyzující reverzibilní přeměnu molekulárního vodíku v některých prokaryotních organismech:\footnote[frame]{\href{https://doi.org/10.1039/C3RA22668A}{Fundamentals and electrochemical applications of [Ni–Fe]-uptake hydrogenases}}
		\item \ce{H2 <=> 2 H+ + 2 e-}
		\item Struktura enzymu obsahuje aktivní místo tvořené ionty Fe a Ni vázanými přes sulfidické můstky.
		\item Železo je stabilně v oxidačním stavu II, redoxních dějů se účastní nikl.
	\end{itemize}
	\begin{figure}
		\adjincludegraphics[height=.3\textheight]{img/Nickel_Iron_Hydrogenase_Active_Site.png}
		\caption*{Aktivní místi NiFe hydrogenasy.\footnote[frame]{Zdroj: \href{https://commons.wikimedia.org/wiki/File:Nickel_Iron_Hydrogenase_Active_Site.png}{CHEM8240edpt/Commons}}}
	\end{figure}
	\vfill
}

\input{../Last}

\end{document}